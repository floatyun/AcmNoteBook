\section{欧拉相关}

\subsection{欧拉函数}
	\par $\phi(n)$表示1到n中与n互素的个数.
	\par 公式 $\phi(n)=n\prod\limits_{p \mid n \& p \in P } 1-\frac{1}{p}$
	\par $\phi(1..n)$总体求解可用线性筛$O(n)$求出.
	\par 单个$\phi(n)$可用分解质因数法直接用公式$O(\sqrt{n}$求出。
	
	%code block
\begin{lstlisting}[language={c++}]
// 时间复杂度sqrt(n)求phi(n) n最大1e12-1e14的级别
// more beautiful version, but slower (just a little bit)
ll phi(ll n) {
	ll a = n;
	for (ll p = 2; p * p <= n; ++p)
		if (!(n % p)) {
			do
				n /= p;
			while (!(n % p));
			a = a / p * (p - 1);
		}
	if (n > 1) a = a / n * (n - 1);  // the rest n is a prime
	return a;
}

// 计算1--n的所有phi(i) 线性时空复杂度,n应该最大是1e7级别的
void get_all_phi(int n, vector<int>& phi) {
	phi.resize(n + 1);
	vector<bool> is_prime(n + 1, true);
	vector<int> prime;
	is_prime[1] = is_prime[0] = false;
	phi[1] = 1;
	for (int i = 2; i <= n; ++i) {
		if (is_prime[i]) {
			prime.push_back(i);
			phi[i] = i - 1;
		}
		for (auto p : prime) {
			if (i * p > n) break;
			is_prime[i * p] = false;
			if (i % p) {
				// i不具有素因子p,i*p对于素因子p来讲次数=1。贡献是(p-1)
				phi[i * p] = phi[i] * (p - 1);
			} else {
				phi[i * p] = phi[i] * p;  // i具有素因子p,则i*p对于欧拉函数值来讲乘以p
				break;  // 保证每个数只被最小素因子访问到。
			}
		}
	}
}
\end{lstlisting}


\subsection{欧拉定理及欧拉降幂公式}
\large{欧拉定理} $$(a,m)=1 \Rightarrow a^{\phi(m)} \equiv 1 \pmod{m} \label{euler1}$$
\par \large{广义欧拉降幂公式}
\begin{equation*}
	a^{B} = 
	\begin{cases}
		a^B                           &\text{if} B < \phi(m);\\
		a^{B \bmod \phi(m) + \phi(m)} &\text{if} B \geq \phi(m) \& (a,m) \neq 1;\\
		a^{B \bmod \phi(m)} &\text{if} B \geq \phi(m) \& (a,m) = 1;\\
	\end{cases}
\end{equation*}
\par 其中第三种情况是可以使用第二种情况的式子,因为欧拉定理——\ref{euler1}。实际上,当模数与底数互素的时候,只需要指数对$\phi{m}$取模降幂即可。
\par 模板略。