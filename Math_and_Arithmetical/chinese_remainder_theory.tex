\subsection{中国剩余定理}
\par \large{结论}
\par 方程组$x \equiv c_i \pmod{m_i} \quad (i=1, 2, 3, \ldots, n)$.其中$m_i$\underline{\textbf{两两互质}}。
\par 中国剩余定理是说,这样的线性同余方程组的通解是$x=x_0+Mt, \, t \in Z$.其中$M=\prod\limits_{i=1}^{n}m_i$,即所有模数的乘积;$$x_0={
	\left(
		\sum\limits_{i=1}^{n}
		c_i M_i {M_i}_{m_i}^{-1} 
	\right) \bmod M
}$$.其中$M_i=\frac{M}{m_i}$,即$M_i$是除掉第$i$个模数$m_i$之外所有模数的积;${M_i}_{m_i}^{-1}$是$M_i$关于模数$m_i$的逆元。\textbf{显然模M意义下,解有且只有一个,即$x_0$。}
\par 复杂度$O(n\log{val})$,对数来源于求逆元。
\par \large{Code}
\begin{lstlisting}[language={c++}]
// m两两互质
// 当模数是long long的时候,两个数相乘要用__int128
// 当模数是__int128的时候,使用ex_gcd解线性同余方程组或者使用快速模乘防止爆__int128
// x是模M范围内的唯一解
bool chinese_remainder_theory(int n, ll c[], ll m[], ll &x, ll &k) {
	ll &M = k;
	ll Mi, inv_Mi;
	__int128 t;
	M = 1; x = 0;
	for (int i= 0; i < n; ++i) M *= m[i];
	for (int i = 0; i < n; ++i) {
		Mi = M/m[i];
		multiplicative_inverse(Mi,m[i],inv_Mi); // 肯定存在
		t = c[i]%M;
		t = t*Mi%M;
		t = t*inv_Mi%M; // 防止爆long long
		x = (x+(ll)t)%M;
	}
	return true;
}
\end{lstlisting}