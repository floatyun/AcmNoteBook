\section{lucas及其拓展}
\subsection{lucas定理}
	\par $$\tbinom{n}{m} \bmod p = \tbinom{\lfloor \frac{n}{p} \rfloor}{\lfloor \frac{m}{p} \rfloor} \tbinom{n \bmod p}{m \bmod p} \bmod p=\tbinom{n/p}{m/p}\tbinom{n\%p}{m\%p} \bmod p$$  
	\par 先预先求出$i! \;(i \in \left[0,p\right))$. 并利用费马小定理和快速幂乘求出每一个$i!$的逆元$(i!)^{-1}$。求$\tbinom{n}{m} \bmod p$,当$m=0$直接就是$1$.若$n,m$都在$p$范围内,则直接转化为$n! \times (m!)^{-1} \times [(n-m)!]^{-1}$.否则就是\textit{lucas定理}缩小规模。
	\par {\bfseries 对一个固定的p,预处理求阶乘及快速模幂求其逆元,时间复杂度$O(p\log_2{p})$。空间复杂度$O(p)$。预处理之后,单次求$\tbinom{n}{m} \bmod p$复杂度$O(\log_{p}{m})$} 
	\begin{lstlisting}[language={c++}]
void prepare(ll p, vector<ll>&fac, vector<ll>&inv_fac) {
	fac.resize(p); inv_fac.resize(p);
	mod_sys mod;
	mod.set_mod(p);
	fac[0] = 1;
	inv_fac[0] = 1;
	for (int i = 1; i < p; ++i) {
		fac[i] = (fac[i-1]*i)%p;
		inv_fac[i] = mod.pow(fac[i], p-2); // 既然能枚举一遍,p*p不应该爆ll
	}
}

// 输入预设0=<n,m<p
inline ll combination(ll n, ll m, ll p, vector<ll>&fac, vector<ll>&inv_fac) {
	if (n < m) return 0;
	return fac[n]*inv_fac[m]%p*inv_fac[n-m]%p;
}

ll lucas(ll n, ll m, ll p, vector<ll>&fac, vector<ll>&inv_fac) {
	if (n < m) return 0;
	ll ans = 1;
	while(true) {
		if (m == 0) return ans;
		if (n < p && m < p) return ans*combination(n,m,p,fac,inv_fac)%p;
		ans = ans * combination(n%p,m%p,p,fac,inv_fac)%p;
		n/=p; m/=p;
	}
}
	\end{lstlisting}

\subsection{拓展lucas的结论}
\par $$\tbinom{n}{m} \equiv
{
	p^{k_1-k_2-k_3} \times b_1 \times {b_2}^{-1} \times {b_3}^{-1} 
	\times c^{u_1-u_2-u_3} 
	\times \frac{{k_1}!}{{k_2}! \times {k_3}!}
} \pmod{p^w}$$.
\begin{enumerate}
	\item 当$r_1 \geq r_2$时,$k_1=k_2+k_3$,故上面这个式子最后分式的部分$\frac{{k_1}!}{{k_2}! \times {k_3}!}=\tbinom{k_1}{k_2}$.
	\item 当$r_1 < r_2$时,$k_1=k_2+k_3+1$,最后的那个分式无法直接变成组合数,但是我们只需要分子分母同时乘以$k_1-k_2$,即可变成组合数。$\frac{{k_1}!}{{k_2}! \times {k_3}!}=(k_1-k_2) \times \tbinom{k_1}{k_2}$
\end{enumerate}

\par 各个字母代表的含义。
\par 与n,m,n-m有关的量k,r,u,v分别用下标1,2,3区分.
\par k,r是除以$p$的商与余数,u,v是除以模数$p^w$的商与余数。
\par b是$n!$($m!$或$(n-m)!$)最后剩下的v个数中不是p的倍数的数的乘积。
\par c是$\left[1,p^w\right]$中不是p的倍数的数的乘积。
\par \textbf{从结论中的式子可以看到b,c我们只关注模$p^k$意义下的值,因此可以预先求出$[1..i]$中不是p的倍数的数的乘积$f(i)$(模$p^k$意义下的)。}
\subsection{$\tbinom{n}{m} \bmod N$的求取}
N是任意正整数。对$N$进行素数分解。$N=\prod\limits_{i=1}^{q}p_i^{k_i}$.
对$\tbinom{n}{m} \bmod p_i^{k_i}$问题,可以通过上一小节的拓展lucas求得,记答案是$c_i$.
于是得到了$q$个线性同余方程,即线性同余方程组$\tbinom{n}{m} \equiv c_i \pmod{p_i^{k_i}} \quad (1 \leq i \leq q)$.
对于线性同余方程组,并且注意到模数$p_i^{k_i}$两两互质,可以用中国剩余定理(也可以用拓欧)解出其通解$x=x_0+kt$。并且由于模数互质,$k=lcm(p_i^{k_i})=N \quad (1 \leq i \leq q)$.所以在$[0,N)$内只有一个特解$x_0$,而这个特解就是$\tbinom{n}{m} \bmod N$.

\begin{lstlisting}[language={c++}]
ll pow(ll a, ll n)
{   
	if (n == 0) return 1;
	// 始终维持要求的数可以表示为(a)^n*t
	ll t = 1;
	while (n > 1)
	{
		if (n&1) t = t*a;
		n >>= 1; a = a*a;
	}
	return a*t; // now n = 1
}

// 极端情况下i*i会爆ll,要改成n开根号(效率低)
// 质因数分解,p_i^{k_i} 共q项 返回q
int factor(ll n, vector<ll>&p, vector<int>&k) {
	p.clear(); k.clear();
	if (n <= 1) return 0;
	int q = 0;
	for (ll i = 2; i*i <= n; ++i) {
		if (!(n%i)) {
			p.push_back(i);
			k.push_back(0);
			do {n /= i; ++k[q];} while (!(n%i));
			++q;
		}
	}
	if (n > 1) {
		p.push_back(n);
		k.push_back(1);
		++q;
	}
	return q;
}

// 求C(n,m)%(p^k)
const ll kMaxPk = 1000000;
// f[i]表示1..i中不是p的倍数的数的乘积(%pk) inv_f则是相应的逆元
ll f[kMaxPk],inv_f[kMaxPk];
ll ex_lucas(ll n, ll m, ll p, ll k, ll pk) {
	ll k1,k2,k3,r1,r2,r3,u1,u2,u3,v1,v2,v3;
	ll ans = 1;
	f[0] = 1; inv_f[0] = 1;
	for (ll j = 1; j < pk; ++j) {
		if (j%p) {
			f[j] = (f[j-1]*j)%pk;
			multiplicative_inverse(f[j],pk,inv_f[j]); // 肯定存在逆元
		} else {
			f[j] = f[j-1];
			inv_f[j] = inv_f[j-1];
		}
	}
	while(1) {
		if (m == 0) return ans;
		k1 = n/p, r1 = n%p;
		k2 = m/p, r2 = m%p;
		k3 = (n-m)/p, r3 = (n-m)%p;
		u1 = n/pk, v1 = n%pk;
		u2 = m/pk, v2 = m%pk;
		u3 = (n-m)/pk, v3 = (n-m)%pk;
		if (k1-k2-k3) { // == 1
			ans = (ans*p)%pk;
		} // else == 0
		ans = (ans*f[v1])%pk;
		ans = (ans*inv_f[v2])%pk;
		ans = (ans*inv_f[v3])%pk;
		if (u1-u2-u3) { // == 1
			ans = (ans*f[pk-1])%pk;
		} // else == 0
		if (r1 < r2) ans = ans*((k1-k2)%pk)%pk;
		n = k1; m = k2;
	}
}

const int kMaxQ = 30;
ll a[kMaxQ];
ll c[kMaxQ];
ll mm[kMaxQ];

// 返回C(n,m)%N N的分解因式后最大的x=p^k
// 可以开x大小的数组
ll ex_lucas_N(ll n, ll m, ll N) {
	vector<ll>p;
	vector<int>k;
	int q = factor(N,p,k);
	for (int i = 0; i < q; ++i) {
		a[i] = 1;
		mm[i] = pow(p[i],k[i]);
		c[i] = ex_lucas(n,m, p[i], k[i], mm[i]);
	}
	ll ans,kk;
	linear_congruence_equations(q,a,c,mm,ans,kk);
	return ans;
}
\end{lstlisting}