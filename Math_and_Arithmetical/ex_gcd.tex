\section{拓展欧几里得算法}
	\par 欧几里得算法直接使用g++中的<algorithm>库中\_\_gcd()函数即可。
	\par $(a,b)=(b,a \bmod b)$.
	\par 拓展欧几里得算法用于求出不定方程$ax+by=(a,b)$的一个特解$x_0,y_0$, 顺带求出$(a,b)$,{\bfseries 通解$x=x_0+\frac{b}{(a,b)}t,\,y=y_0-\frac{a}{(a,b)}t \; (t \in Z)$}.
	\subsection{解不定方程}
		\par 不定方程$ax+by=c$有解等价于$(a,b) \mid c$.据此判断是否有解,若有解,则原不定方程的解是(且只能是)欧几里得算法所解不定方程解的$\frac{c}{(a,b)}$倍。
	\subsection{求解模线性方程(线性同余方程)}
		$ax \equiv c\pmod{m} \Longleftrightarrow ax+my=c$.
	\subsection{求乘法逆元}
		\par $ab \equiv 1 \pmod{m}$, 则a关于模m的乘法逆元是b,b关于模m的乘法逆元是a。或者说$Z_m$群中a和b互为乘法逆元。
		\par 用乘法逆元有$\frac{A}{b} \equiv A \times b^{-1} \pmod{c}$.当左边的式子A是很大的数,而b是小规模数,且除出来的数一定是整数的时候,可以用右式边算边模。
		\par \textbf{求解} $ax \equiv 1 \pmod{m} \Longleftrightarrow ax+my= 1$.解出的x即为解,只是注意需要用通解公式将$x$调整到$Z_m$范围内。
	\subsection{线性同余方程组}
		\par 方程组$a_ix \equiv c_i \pmod{m_i} \quad (i=1, 2, 3, \ldots, n)$
		\par 对每一个现行同余方程,若无解,则方程组无解;否则,可以解得$x=k_it_i+x_i, \; t_i \in Z$.而$x_i \in \left[0,k_i\right), k_i = \frac{m_i}{(a_i,m_i)} \leq m_i$.不妨增加$x_0=0,k_0=1$,即增加式子$x=t_0+0$.
		\par 现在考虑同时满足$x=k_1t_1+x_1$与$x=k_2t_2+x_2$两个约束的x能否合并成一个依旧如此形式的一个表达式,即$x=k_0t+x_0$.
		\par 联立两个方程,易得$k_1t_1-k_2t_2=x_2-x_1$,将其视作关于$t_1,t_2$的不定方程。若这个方程无解,说明同时满足两个条件的x不存在;否则,每确定一个$t_1$可以代入$x=k_1t_1+x_1$确定一个$x$. 如果我们只是解出$t_1$的话,不妨换成$k_1t_1+k_2t_2=x_2-x_1$,$t_1$的每一个解是不变的。
		\par 假设$k_1t_1+k_2t_2=x_2-x_1$有解,并且解为$t_1=t_{1_0}+\frac{k_2}{(k_1,k_2)}t \quad t_{1_0} \in \left[ 0,\frac{k_2}{(k_1,k_2)} \right), \; t \in Z$.
		\par 代入$x=k_1t_1+x_1$得$x=k_1(t_{1_0}+\frac{k_2}{(k_1,k_2)}t)+x_1=\frac{k_1k_2}{(k_1,k_2)}t+(k_1t_{1_0}+x_1)$.而$k_1t_{1_0}+x_1 < \frac{k_1k_2}{(k_1,k_2)} \Longleftrightarrow t_{1_0}+\frac{x_1}{k_1} < \frac{k_2}{(k_1,k_2)}$.而$t_{1_0} < \frac{k_2}{(k_1,k_2)}$.注意到$t_{1_0},\frac{k_2}{(k_1,k_2)}$是整数,故$t_{1_0} \leq \frac{k_2}{(k_1,k_2)}-1$.又$\frac{x_1}{k_1}\, <\, 1$.这两个不等式相加即可得到$$t_{1_0}+\frac{x_1}{k_1} < \frac{k_2}{(k_1,k_2)}$$。即符合前面定义的形式:$$(k_1t_{1_0}+x_1) \in \left[0, \frac{k_1k_2}{(k_1,k_2)}\right)$$.
		\par 合并为$x=kt+x_0.$的形式有:\begin{align*}
			k &=\frac{k_1,k_2}{(k_1,k_2)}=[k_1,k_2] \\
			x_0 &=k_1t_{1_0}+x_1
		\end{align*}
		\par 对于写程序,由于我们引入了$x=x_0=0,k=k_0=1$.可以每次$x_0,k_0$与$x_i,k_i$合并成$x_0,k_0$.故程序迭代是$x+=kt,k=[k,k_i]$,其中$t_0$是$k_0t_0+k_it_i=x_i-x_0$的不定方程的最小非负整数解。
		\subsubsection{程序设计方面爆long long的问题及应对策略}
		\par {\bfseries 由于k的迭代是不断求最小公倍数,而特解x始终是小于k的,因此k和x可能会增长的很快导致爆long long.尤其是k很容易爆掉。}
		\par 如何解决爆炸的问题?或许可以用{\textbf \_\_int128}。如果k和x还是都爆掉了,那么没法子,只能设法自己实现k和x的存储,注意到解不定方程需要求$(k_0,k_i)$与$\frac{k_i}{(k_0,k_i)}$,所以要大数加减,大数乘除模普通数的实现。估计够呛。
		\begin{lstlisting}[language={c++}]
// poj 2891 然而poj不支持__int128和C++11
#include <bits/stdc++.h>
typedef __int128 ll;

// 求解不定方程ax+by=(a,b)的一组特解并返回a,b最大公约数
// x,y存储返回的一组特解。易懂version
ll ex_gcd1(ll a, ll b, ll &x, ll &y) {
	if (b) {
		auto d = ex_gcd1(b, a%b, x,  y);
		auto x_bac = x;
		x = y; // x设为后
		y = x_bac - a/b * y; // y设为前-a/b*后
		return d;
	} else {
		x = 1; y = 0;
		return a;
	}
}

// 求解不定方程ax+by=(a,b)的一组特解并返回a,b最大公约数
// x,y存储返回的一组特解。
ll ex_gcd(ll a, ll b, ll &x, ll &y) {
	if (b) {
		auto d = ex_gcd(b, a%b, y, x); // 注意x和y位置互换了。
		// x是后,无需赋值,y是 前-a/b*后 即 y -= a/b*x
		y -= a/b*x;
		return d;
	} else {
		x = 1; y = 0;
		return a;
	}
}

// 求解不定方程ax+by=c.
// 返回值表示是否有解
// d存储是(a,b)
// 当有解的情况下
// x,y存储一组特解,并且确保x是最小的非负整数。
// 通解是X=x+(b/d)*t,Y=y-(a/d)*t t是整数。
bool binary_linear_indefinite_equation(ll a, ll b, ll c, ll &x, ll &y, ll &d) {
	d = ex_gcd(a, b, x, y); // solve: ax+by=(a,b)
	if (c%d) return false;
	x *= c/d;
	y *= c/d;
	auto k = b/d;
	x = (x%k+k)%k; // 调为最小非负整数
	y = (d-a*x)/b;
	return true;
}

// 线性同余方程 Linear congruence equation
// ax = c (mod m) <===> ax+my=c
// x存储最小非负解,通解X=x+kt t为整数
// 有解的情况下,最小非负解x肯定在[0,m)范围内
bool linear_congruence_equation(ll a, ll c, ll m, ll &x, ll &k) {
	ll y, d;
	auto ans = binary_linear_indefinite_equation(a, m, c, x, y, d);
	k = m/d;
	return ans;
}

// 求a在Zm<+,*>中的乘法逆元x
// 返回逆元是否存在,x存储逆元
// ax = 1 (mod m)
bool multiplicative_inverse(ll a, ll m, ll &x) {
	ll k;
	return linear_congruence_equation(a, 1, m, x, k);
	// assert(k == m);
}

// 线性同余方程组 Linear congruence equations
// a_ix = c_i (mod m_i) 共n个
// 可能存在的问题,由于迭代过程中k一直在求最小公倍数,所以可能会爆long long,这个,最佳的方法是直接暴力把ll的定义改为__int128
// 但是要注意__int128的输入输出
// 如果还是爆,我没法子了
bool linear_congruence_equations(int n, ll a[], ll c[], ll m[], ll &x, ll &k) {
	ll x_i, k_i, t, t_i, d;
	x = 0; k = 1;
	for (int i = 0; i < n; ++i) {
		if (!linear_congruence_equation(a[i], c[i], m[i], x_i, k_i))
			return false; 
		// kt+x
		// k_it_i+x_i
		if (!binary_linear_indefinite_equation(
			k, k_i, x_i-x, t, t_i, d))
			return false;
		x += k*t;
		k *= k_i/d;
	}
	return true;
}

inline ll read()
{
	ll x = 0;
	bool f = 0;
	char ch = getchar();
	while (ch < '0' || '9' < ch)
		f |= ch == '-', ch = getchar();
	while ('0' <= ch && ch <= '9')
		x = x * 10 + ch - '0', ch = getchar();
	return f ? -x : x;
}

void write(ll a)
{
	if (a < 0)
	{
		putchar('-');
		a = -a;
	}
	if (a >= 10)
	{
		write(a / 10);
	}
	putchar(a % 10 + '0');
}

const int kMaxN = 10000;
ll a[kMaxN]; // ax=c (mod c)
ll m[kMaxN];
ll c[kMaxN]; 
void solve(int n) {
	for (int i = 0; i < n; ++i) {
		a[i] = 1;
		m[i] = read();
		c[i] = read();
	}
	ll x,k;
	auto ans = linear_congruence_equations(n,a,c,m,x,k);
	if (ans) {
		write(x);
		putchar('\n');
	} else
		puts("-1");
}

int main()
{
	int n;
	while (scanf("%d",&n) != EOF) {
		solve(n);
	}
}
		\end{lstlisting}