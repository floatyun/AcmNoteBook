\section{lucas定理}
    \par $$\tbinom{n}{m} \bmod p = \tbinom{\lfloor \frac{n}{p} \rfloor}{\lfloor \frac{m}{p} \rfloor} \tbinom{n \bmod p}{m \bmod p} \bmod p=\tbinom{n/p}{m/p}\tbinom{n\%p}{m\%p} \bmod p$$  
    \par 先预先求出$i! \;(i \in \left[0,p\right))$. 并利用费马小定理和快速幂乘求出每一个$i!$的逆元$(i!)^{-1}$。求$\tbinom{n}{m} \bmod p$,当$m=0$直接就是$1$.若$n,m$都在$p$范围内,则直接转化为$n! \times (m!)^{-1} \times [(n-m)!]^{-1}$.否则就是\textit{lucas定理}缩小规模。
    \par {\bfseries 对一个固定的p,预处理求阶乘及快速模幂求其逆元,时间复杂度$O(p\log_2{p})$。空间复杂度$O(p)$。预处理之后,单次求$\tbinom{n}{m} \bmod p$复杂度$O(\log_{p}{m})$} 
    \begin{lstlisting}[language={c++}]
void prepare(ll p, vector<ll>&fac, vector<ll>&inv_fac) {
    fac.resize(p); inv_fac.resize(p);
    mod_sys mod;
    mod.set_mod(p);
    fac[0] = 1;
    inv_fac[0] = 1;
    for (int i = 1; i < p; ++i) {
        fac[i] = (fac[i-1]*i)%p;
        inv_fac[i] = mod.pow(fac[i], p-2); // 既然能枚举一遍,p*p不应该爆ll
    }
}

// 输入预设0=<n,m<p
inline ll combination(ll n, ll m, ll p, vector<ll>&fac, vector<ll>&inv_fac) {
    if (n < m) return 0;
    return fac[n]*inv_fac[m]%p*inv_fac[n-m]%p;
}

ll lucas(ll n, ll m, ll p, vector<ll>&fac, vector<ll>&inv_fac) {
    if (n < m) return 0;
    ll ans = 1;
    while(true) {
        if (m == 0) return ans;
        if (n < p && m < p) return ans*combination(n,m,p,fac,inv_fac)%p;
        ans = ans * combination(n%p,m%p,p,fac,inv_fac)%p;
        n/=p; m/=p;
    }
}
    \end{lstlisting}