\subsection{Miller-Rabin素数检测算法}
其基于以下两个定理。
\begin{enumerate}
    \item Fermat小定理 若n是素数,则$\forall a(a \not\equiv 0 \pmod{n})$,有$a^{n-1} \equiv 1 \pmod{n}$.
    \item 二次探测定理 若n是素数,则$x^2 \equiv 1 \pmod{n}$只有平凡根$x=\pm1$,即$x=1,x=n-1$.
\end{enumerate}
\subsubsection{具体算法}
假设$n$是奇数,令$n=m \times 2^q (q \geq 1)$,其中$m$是奇数.
对于序列$a^m \bmod n, a^{2m} \bmod n, a^{4m} \bmod n,\ldots,a^{2^q \times m} \bmod n$.
最后一项就是费马小定理中的$a^{n-1}$, 并且每一项都是前一项的平方。
我们一项一项往后计算。
\begin{itemize}
    \item 若当前项为1,后面每一项显然都是1。而根据二次探测定理,n是素数必须前面一项是1或n-1.如果不符合,断言不是素数;符合,断言是素数。
    \item 若当前项不是1,暂时不断言,接着往后算。除非当前是最后一项了,那么断言不是素数。
\end{itemize}
\textbf{当然,如果第一项是1,由于不存在二次探测的方程,所以不检验前面一项(或者认为前面一项符合条件)。}
\subsubsection{Code}
使用了快速幂模和快速幂加模板mod\_sys。下面代码只是miller-rabin核心代码。
\begin{lstlisting}[language={c++}]
// 如果只是int范围内,可以将pow_v2改为pow,mlt改为普通乘法
bool miller_rabin(ll a, ll n, ll q, ll m, mod_sys& mod) {
    a = mod.pow_v2(a, m);
    bool is_ordinary = true;
    for (int i = 0; i < q; ++i) {
        if (a == 1) {
            return is_ordinary;
        } else {
            is_ordinary = (a == n-1);
            a = mod.mlt(a,a);
        }
    }
    return (a==1)&&(is_ordinary); // 最后一项
}

// 使用miller_rabin检测是否是素数
const int kCheckCnt = 8;
// 为了随机数
random_device rd;
mt19937_64 gen(rd());
bool miller_rabin(ll n) {
    if (n == 2) return true;
    if ((n <= 2) || (n&1^1)) return false;
    // 2^q×m表示原本输入的n-1
    ll m = n, q = 0;
    do { m >>= 1; ++q; } while(m&1^1);
    // 随机数生成,[1,n-1] 均匀分布
    uniform_int_distribution<> dis(1, n-1);
    mod_sys mod;
    mod.set_mod(n);
    for (int i = 0; i < kCheckCnt; ++i)
        if (!miller_rabin(dis(gen), n, q, m, mod))
            return false;
    return true;
}
\end{lstlisting}