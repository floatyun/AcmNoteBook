\section{二次剩余}
\par $p$是奇素数。所有的运算都是在群$Z_{p}^{*}$中的运算。方程$x^2=a \neq 0$问是否有解,以及解是什么?若有解,$a$就是模$p$的二次剩余;若无解,则$a$就是模$p$的非二次剩余。
\begin{enumerate}
	\item $a=0$,显然只有唯一解$x=0$.
	\item $a\neq 0$,有解等价于$a^{\frac{p-1}{2}}=1$;无解等价于$a^{\frac{p-1}{2}}=-1$.
\end{enumerate}
\par 群$Z_{p}^{*}$恰好有一半的元素是二次剩余,一半的元素不是二次剩余。当元素$a$是二次剩余是,解有且只有两个$x_0,x_1$,且$x_0=-x_1$.
\par 因此,验证$a$是否是二次剩余可以用快速模幂,复杂度$O(\log_2{p})$.
\par 对于二次剩余$a$,求解x使用\underline{Cipolla}(洋葱? 奇波拉?)算法。
\subsection{Cipolla洋葱算法}
\par 这是一个随机性算法,复杂度是$O(\log_2{p})$
\par {\bfseries 这个算法是在域$F_{p^2}$上进行运算的,在这个域上做乘法、幂运算,然后解的那个表达式算出来之后,一定是属于域$F_{p}$的,$F_{p^2}$是$F_{p}$的扩充。}
\begin{enumerate}
	\item 使用随机的方法,找到一个满足$b$满足$b^2-a$不是二次剩余。需要检验$b^2-a$是不是二次剩余的期望次数是$2$.每次检验是$O(\log_2{p})$.
	\item 定义域$F_{p^2}$上的\textit{"虚部单位"}为$\omega$,其中$F_{p^2}={a+b\omega| a,b \in F_p}$,且$\omega^2=b^2-a$.
	\item $x^2=a$的其中一个{\bfseries 解是$$x=(b+\omega)^{\frac{p+1}{2}}.$$}注意,虽然运算是$F_{p^2}$上的,但是右边的那个式子算出来的结果刚好是\textit{"虚部"}为0的元素。原因是$x^2=a \quad (x \in F_p)$与$x^2=(x_a+x_b\omega)^2=a+0\omega  \quad  (x \in F_{p^2},x_a,x_b \in F_p)$这两个方程都有且仅有两个解。而前面方程的解显然是后面两个方程的解,所以前面方程的解。所以前面这个方程和后面这个方程的解是完全一样的。而对于后面这个方程$x=(b+\omega)^{\frac{p+1}{2}}$代入后面的方程去验证(为了计算简洁,先证明了一些性质),会发现它的确是解,于是这个式子就是一开始方程的解。
\end{enumerate}
\par 下面为模板:
\subsection{高效大范围随机数产生方法}
\par 采用c++11的随机数生成法,更加简单,而且可以产生int或ll范围内的整数。而rand()只能产生16位short的整数,放大会有锯齿。
\begin{lstlisting}[language={c++}]
// 为了随机数
random_device rd;
mt19937_64 gen(rd()); // 64代表是64为,根据需要可以改小
// uniform_int_distribution<> dis(1, n-1); //[1,n-1]均匀分布随机数 设置随机数范围
// dis(gen) 返回一个随机数
\end{lstlisting}

\subsection{域$F_{p^2}$乘法与幂运算模板}
\par 应当放在\textbf{mod\_sys(见快速乘幂)}的模板前面。
\begin{lstlisting}[language={c++}]
	// 二次剩余模板,注意,各个函数都假设p^2即mod^2不会爆ll
	// 域F_p的拓展域F_{p^2}
	struct F_p2_node{ll a,b;}; // F_p2_node所有的运算都应该通过F_p2_sys来管理调用。
	// 预设所有传入参数都是合法的。
	struct F_p2_sys{
		ll p,w2;
		typedef F_p2_node nd;
		nd mlt(const nd&x, const nd&y) const {
			nd ans;
			//(x.a+x.bw)(y.a+y.bw)
			ans.a = (x.a*y.a%p+x.b*y.b%p*w2%p)%p;
			ans.b = (x.b*y.a%p+x.a*y.b%p)%p;
			return ans;
		}
		// pre n>=0 非递归版 不可改成引用!
		nd pow(nd a, ll n) const {
			if (n == 0) {return nd{1,0};}
			// 始终维持要求的数可以表示为(a)^n*t
			nd t{1,0};
			while (n > 1)
			{
				if (n&1) t = mlt(t,a);
				n >>= 1; a = mlt(a,a);
			}
			return mlt(t,a);
		}
	};
\end{lstlisting}
\subsection{Cipolla洋葱算法实现模板}
\textbf{需要使用先用快速乘幂的模板中的mod\_sys,实际上,这个模板只是给那个mod\_sys增加了成员函数。}
\begin{lstlisting}[language={c++}]
struct mod_sys{
	// template code from 快速乘幂 见前面章节
	some code....

	// 预设a \not\equiv 0 (mod mod) 返回a是否是二次剩余
	// 预设mod是奇素数
	// 预设p^2不会爆long long,使用pow 若爆则改用pow_v2或者__int128
	bool is_quadratic_residue(ll a) {
		return 1 == pow(a,mod-1>>1);
	}

	// 解方程x^2 = a (mod mod)  p = mod 是奇质数
	// 返回是否有解。如果有x0,x1(x0<=x1)存储解 仅解是0的时候取等号
	// 洋葱算法。
	// 预设p^2不会爆long long,使用pow 若爆则改用pow_v2或者__int128
	bool sqrt(ll a,ll& x0, ll& x1) { // 需要一个随机数生成器
		a = to_std(a);
		if (a == 0) {x0 = x1 = 0; return true;}
		if (!is_quadratic_residue(a)) return false;
		uniform_int_distribution<> dis(1, mod-1);
		ll b,w2;
		while(true) {
			b = dis(gen);
			w2 = to_std((b*b)%mod-a);
			if (!is_quadratic_residue(w2)) break;
		}
		// x = (b+w)^{(p+1)/2}
		F_p2_sys fp2{mod,w2}; // p,w2
		F_p2_node t{b,1}; // a,b(means a+bw)
		auto x = fp2.pow(t,mod+1>>1);
		x0 = to_std(x.a); // assert(x.b==0);
		x1 = mod-x.a;
		if (x0 > x1) swap(x0,x1);
		return true;
	}
};
\end{lstlisting}