\section{莫比乌斯反演}
		\par $F$是已知函数,$f$是未知函数。$\mu(x)$是固定函数,即\textbf{莫比乌斯函数}。
		\subsection{莫比乌斯函数$\mu(x)$}
			\begin{itemize}
				\item $\mu(1)=1$
				\item x为不同的质数的乘积。若质数个数为奇数,则$\mu(x)=-1$; 偶数个$\mu(x)=1$.
				\item 剩下的情况,即x的某个素因子的次数大于等于2. $\mu(x)=0$.
			\end{itemize}
			\par 莫比乌斯函是积性函数,$10^7$规模的数据,故可用\textbf{线性筛思想}解决,否则需要使用杜教筛。
			%code block
			\begin{lstlisting}[language={c++}]
// 计算所有的\mu(x) x in [1..n]. 线性复杂度。
void get_all_mu(int n, vector<int>&mu) {
  mu.resize(n+1);
  vector<bool>is_prime(n+1, true);
  vector<int>prime;
  is_prime[1] = is_prime[0] = false; mu[1] = 1;
  for (int i = 2; i <= n; ++i) {
    if (is_prime[i]) {
      prime.push_back(i);
      mu[i] = -1;
    }
    for (auto p : prime) {
      if (i*p > n) break;
      is_prime[i*p] = false;
      if (i%p) {
        mu[i*p] = -mu[i]; // i不具有素因子p,具有积性
      } else {
        mu[i*p] = 0; // i具有素因子p,则i*p具有素因子的平方的因子
        break; // 保证每个数只被最小素因子访问到。
      }
    }
  }
}
			\end{lstlisting}

		\subsection{整除分块}
		基本形式$\sum\limits_{i=1}^{n} \lfloor \frac{n}{i} \rfloor$.假设n的因数从小到大保存在d[1..k]中。
		\begin{equation}
			\mathbf{d} = 
			\left(
				\begin{array}{ccc}
					d_1, d_2,  d_3,  \ldots,  d_k
				\end{array}	
			\right)
		\end{equation}
		显然$i \in (d_j,d_{j+1}]$时$\lfloor \frac{n}{i} \rfloor$结果相同,假设结果为$k$, 则$d_{j+1}=\lfloor \frac{n}{k} \rfloor$。
		因此可以一段一段区间的跳跃求和,将复杂度降到$\sqrt{n}$.
		\par 大多数时候不是基本形式,而是基本形式乘以一个函数求和。这个时候,可以对这个函数求前缀和,然后用整除分块的代码做一下变通即可。
		
		%code block
		\begin{lstlisting}[language={c++}]
for(int l=1,r;l<=n;l=r+1) {
  r=n/(n/l);
  ans+=(r-l+1)*(n/l)*(sum[r]-sum[l-1]); // sum是乘以函数的前缀和 
}
		\end{lstlisting}

		\subsection{莫比乌斯反演定理}
			\subsubsection{因数形式}
			\par d取遍n的所有因数。$F(n)=\sum\limits_{d \mid n}f(d)$,反演求出未知的$f$的表达式为$f(n)=\sum\limits_{d \mid n}\mu(d)F(\frac{n}{d})=\sum\limits_{d \mid n}F(d)\mu(\frac{n}{d})$
			\subsubsection{倍数形式}
			\par d取遍n的所有倍数。$F(n)=\sum\limits_{n \mid d}{f(d)}$,反演求出未知的$f$的表达式为$f(n)=\sum\limits_{n \mid d}{\mu(\frac{d}{n})F(d)}$.
		\subsection{做题思路或者技巧}
			\begin{enumerate}
				\item 构造能够直接写出表达式的已知函数F,然后尝试因数或者倍数形式的小f.
				\item 利用反演定理求出f的表达式。
				\item 将ans用小f的形式表达出来。可能还是一个关于f的$\Sigma$求和式。
				\item 代入f的反演结果,根据数学尝试变换枚举变量的顺序。
				\item 与gcd有关的莫比乌斯反演。一般我们都是套路的去设$f(d)$\mbox{为}$gcd(i,j)=d$\mbox{的个数},$F(n)$\mbox{为}$gcd(i,j)=d\mbox{的倍数}$的个数。
				\item 注意最后的式子是不是包含一个整除分块的部分。
			\end{enumerate}

			%code block
			\begin{lstlisting}[language={c++}]
			\end{lstlisting}
