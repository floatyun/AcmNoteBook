%!TEX program = xelatex
\documentclass{article}
%\documentclass[utf8]{ctexart}
\usepackage[UTF8]{ctex}
\usepackage{fontspec}
\usepackage{listings}
\usepackage{xeCJK}

\usepackage{xparse, mathtools}
\usepackage{amsmath}
\usepackage{geometry}%页边距

%设置也的上下左右边距
\geometry{left=3.17cm,right=3.17cm,top=2.54cm,bottom=2.54cm}

\usepackage[dvipsnames, svgnames, x11names]{xcolor} % 定制颜色
\definecolor{mygreen}{rgb}{0,0.6,0}
\definecolor{mygray}{rgb}{0.5,0.5,0.5}
\definecolor{mymauve}{rgb}{0.58,0,0.82}

%行向量
\NewDocumentCommand{\Rowvec}{ O{,} m }
{
	\vector_main:nnnn { p } { & } { #1 } { #2 }
}
\NewDocumentCommand{\Colvec}{ O{,} m }
{
	\vector_main:nnnn { p } { \\ } { #1 } { #2 }
}

\setmainfont{Candara}[BoldFont={Microsoft JhengHei}, ItalicFont={Corbel}]
%\setCJKmainfont{STKaiti}[BoldFont={FandolHei-Bold}, ItalicFont={STXingkai}]
\setCJKmainfont{FZYBKSJW.TTF}[BoldFont={FandolHei-Bold.otf}, ItalicFont={STXingkai.ttf}] %正文字体
%[BoldFont={FandolHei-Bold}, ItalicFont={STXingkai}]1
\setmonofont{Consolas} %inconsolata
\newcommand{\shusong}{\CJKfontspec{FZShuSong-Z01S}} %方正书宋
\newcommand{\xingkai}{\CJKfontspec{STXingkai}}%华文行楷
\usepackage{indentfirst}%设置缩进
\setlength{\parindent}{2em} %也可在正文中非全局设置



\lstset{ %
	backgroundcolor=\color{white},      % choose the background color
	basicstyle=\small\ttfamily,  % size of fonts used for the code
	numbers=none,                                        % 在左侧显示行号
  numberstyle=\small\color{Dandelion},                       % 设定行号格式
	columns=fixed,
	tabsize=2,
	breaklines=true,               % automatic line breaking only at whitespace
	captionpos=b,                  % sets the caption-position to bottom
	commentstyle=\color{mygreen},  % comment style
	escapeinside={\%*}{*)},        % if you want to add LaTeX within your code
	keywordstyle=\color{blue},     % keyword style
	stringstyle=\color{mymauve}\ttfamily,  % string literal style
	frame=double,
	rulesepcolor=\color{red!20!green!20!blue!20},
	identifierstyle=\color{VioletRed},
	morekeywords={ll},
	language=c++,
}



\begin{document}
	\subsection{中国剩余定理}
\par \large{结论}
\par 方程组$x \equiv c_i \pmod{m_i} \quad (i=1, 2, 3, \ldots, n)$.其中$m_i$\underline{\textbf{两两互质}}。
\par 中国剩余定理是说,这样的线性同余方程组的通解是$x=x_0+Mt, \, t \in Z$.其中$M=\prod\limits_{i=1}^{n}m_i$,即所有模数的乘积;$$x_0={
	\left(
		\sum\limits_{i=1}^{n}
		c_i M_i {M_i}_{m_i}^{-1} 
	\right) \bmod M
}$$.其中$M_i=\frac{M}{m_i}$,即$M_i$是除掉第$i$个模数$m_i$之外所有模数的积;${M_i}_{m_i}^{-1}$是$M_i$关于模数$m_i$的逆元。\textbf{显然模M意义下,解有且只有一个,即$x_0$。}
\par 复杂度$O(n\log{val})$,对数来源于求逆元。
\par \large{Code}
\begin{lstlisting}[language={c++}]
// m两两互质
// 当模数是long long的时候,两个数相乘要用__int128
// 当模数是__int128的时候,使用ex_gcd解线性同余方程组或者使用快速模乘防止爆__int128
// x是模M范围内的唯一解
bool chinese_remainder_theory(int n, ll c[], ll m[], ll &x, ll &k) {
	ll &M = k;
	ll Mi, inv_Mi;
	__int128 t;
	M = 1; x = 0;
	for (int i= 0; i < n; ++i) M *= m[i];
	for (int i = 0; i < n; ++i) {
		Mi = M/m[i];
		multiplicative_inverse(Mi,m[i],inv_Mi); // 肯定存在
		t = c[i]%M;
		t = t*Mi%M;
		t = t*inv_Mi%M; // 防止爆long long
		x = (x+(ll)t)%M;
	}
	return true;
}
\end{lstlisting}
\end{document}
