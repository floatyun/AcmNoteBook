%!TEX program = xelatex
\documentclass{article}
\usepackage[UTF8]{ctex}
\usepackage{fontspec}
\usepackage{listings}
\usepackage{xeCJK}
\setCJKmainfont{STKaiti}%正文字体
\newcommand{\shusong}{\CJKfontspec{FZShuSong-Z01S}}%方正书宋
\newcommand{\xingkai}{\CJKfontspec{STXingkai}}%华文行楷
\usepackage{indentfirst}%设置缩进
\setlength{\parindent}{2em} %也可在正文中非全局设置
\setmainfont{Consolas}

\begin{document}


% 目录区
\tableofcontents
% 目录区


\section{计算几何}
\label{sec:geometry}

\subsection{几何基础}
\label{sec:geometry_basic}
\subsubsection{点积}
\label{sec:dianji}
点积的应用  

\begin{enumerate}
\item 判断两个向量是否垂直$a\bot b <=> a·b = 0$
\item 求两个向量的夹角,点积$<0$为钝角,点积$>0$为锐角
\end{enumerate}

\begin{lstlisting}[language={c}]
    //点积
    double dot(vector a,vector b){
        return a.x*b.x+a.y*b.y;
    }
    //求夹角
    double Angle(vector a,vector b){
        return acos(dot(a,b)/len(a)/len(b));
    }
    //求模长
    double len(vector a){ 
        return sqrt(dot(a,a));
    }
\end{lstlisting}

\subsubsection{叉积}
\label{sec:chaji}

\begin{enumerate}
    \item 判断平行$a\times b = 0$
    \item 判断左右$a\times b > 0$ 在左边,$< 0$在右边
\end{enumerate}

\subsubsection{点和直线}

直线上所有的点表示为$P = P_0+tv$。若已知直线的两个点A、B,则方程为$A+(B-A)t$

\begin{lstlisting}[language={c}]
    //点到直线的距离
    //利用叉积求面积,然后除以平行四边形的底边长,得到平行四边形的高即点到直线的距离
    double distl(point p,point a,point b)
    {
    	vector v=p-a; vector u=b-a;
    	return fabs(cross(v,u))/len(u);
    }

    //点到线段的距离
    //比点到直线的距离稍微复杂。因为是线段,所以如果平行四边形的高在区域之外的话就不合理,这时候需要计算点到距离较近的端点的距离
    double dists(point p,point a,point b)
    {
	    if (a==b) return len(p-a);
	    vector v1=b-a,v2=p-a,v3=p-b;
	    if (dcmp(dot(v1,v2))<0) return len(v2);
	    else if (dcmp(dot(v1,v3))>0) return len(v3);
	    return fabs(cross(v1,v2))/len(v1);
    }

    //判断两个线段是否相交
    //跨立实验:判断一条线段的两端是否在另一条线段的两侧(两个端点与另一线段的叉积乘积为负)。
    //需要正反判断两侧。
    bool segment(point a,point b,point c,point d)
    {
	    double c1=cross(b-a,c-a),c2=cross(b-a,d-a);
	    double d1=cross(d-c,a-c),d2=cross(d-c,b-c);
	    return dcmp(c1)*dcmp(c2)<0&&dcmp(d1)*dcmp(d2)<0;
    }

\end{lstlisting}
\section{动态规划}
		\subsection{背包问题}
			\subsubsection{0-1背包}
				\par n个物体,每个物体的费用为$cost_i$,价值为$val_i$.最大允许费用为$max\_cost$,问能挑选出来的最大价值是什么。
				\begin{lstlisting}[language={c++}]
for (int i = 1; i <= max_cost; i++)
	for (int l = max_cost - i; l >= 0; l--)
		f[l + i] = max(f[l] + cost[i], f[l + i]);		
				\end{lstlisting}
\end{document}