\section{线段树}
	\par 采用左闭右开区间模式。数组元素、线段树节点下标通通从0开始记。$p=(ch-1)/2, l=2p+1, r=2p+2.$
	\par 元素个数为$item\_sz$, 线段树节点数组大小$seg\_nd\_sz=2^{{\lceil \log_2{item\_sz}  \rceil} + 1}-1 < 2^{\log_2{item\_sz} + 1 + 1} = 4 \times item\_sz$. 放缩操作是向上取整的结果肯定小于直接加1的结果,并且还没有减去最后需要减去的1.\textit{所以,简单奢侈的方式是直接开4倍。}
	\par lazy标记,就是记录区间中元素共有的操作。节点的最值、和等标记是不考虑节点自身的lazy标记的结果,但是这些标记由子节点求解的时候需要考虑子节点的lazy标记;但是对于最值、和等的询问,返回的值是考虑了lazy标记的结果。
	\par 即\textit{ mx,mn,sum标记的值和get\_min,get\_mn,get\_sum的返回的值不一定相同}。
	\subsection{区间集体加+查询区间和最大最小值模板}

\begin{lstlisting}[language={c++}]
#include <bits/stdc++.h>
using namespace std;
typedef long long ll;

namespace lly {
// 所有下标符合C++风格
// 奢侈做法:使用全局变量开4倍数组。
// 最小值,最大值,区间和,区间加相同数模板
// 考场抄代码所有item_type, sum_type 换成ll即可
struct segment_tree {
	// 最大的初始数组大小和响应的线段树节点数组最大大小。
	const static int kMaxItemSize = 100000;
	const static int kMaxSegTreeSize = kMaxItemSize << 2;

	int item_sz;
	int seg_sz;
	int &n = item_sz;
	int &stn = seg_sz;

	typedef ll item_type;
	typedef ll sum_type;

	struct nd {
		int l, r;
		item_type mx;  // max
		item_type mn;  // min
		sum_type sm;   // sum flag
		// lazy flags
		item_type all_add;  // lazy标记,表示区间内的数都要加上的数
		// other flags
		inline void add(item_type a) {
			all_add += a;
			mx += a;
			mn += a;
			sm += (sum_type)(r-l)*a;
		}
		inline int mid() { return l + (r - l) / 2; }
	};

	item_type a[kMaxItemSize];
	nd nds[kMaxSegTreeSize];

	void init(int cnt) {  // 屏幕输入数组a版本
		n = cnt;
		for (int i = 0; i < n; ++i) scanf("%lld", a+i);//cin >> a[i];
		seg_sz = (2 << (int)(ceil( log2(item_sz) ))) - 1;
	}

	void init(item_type src[], int cnt) {  // 内存数组输入数组a版本
		n = cnt;
		for (int i = 0; i < n; ++i) a[i] = src[i];
		seg_sz = (2 << (int)(ceil( log2(item_sz) ))) - 1;
	}

	inline int parent(int x) { return (x - 1) >> 1; }
	inline int lchild(int x) { return (x << 1) | 1; }
	inline int rchild(int x) { return (x << 1) + 2; }

	inline void build() { build(0, n, 0); }

	inline void set_flags(int root, int i) {  // nds[root]用a[i]设置各类标志
		auto &p = nds[root];
		p.l = i;
		p.r = i + 1;
		p.mx = a[i];
		p.mn = a[i];
		p.sm = a[i];
	}

	inline void merge_flags(int root) {
		auto &p = nds[root];
		auto &l = nds[lchild(root)];
		auto &r = nds[rchild(root)];
		p.l = l.l;
		p.r = r.r;
		p.mx = max(l.mx, r.mx);
		p.mn = min(l.mn, r.mn);
		p.sm = l.sm + r.sm + p.all_add * (sum_type)(p.r - p.l);
	}

	void build(int l, int r, int root) {
		nds[root].all_add = 0;
		if (l + 1 == r) {
			set_flags(root, l);
			return;
		}
		int m = l + (r - l) / 2;
		build(l, m, lchild(root));
		build(m, r, rchild(root));
		merge_flags(root);
	}

	// [l,r)区间的数都加上val
	void add(int l, int r, item_type val, int root = 0) {
		if (l == nds[root].l && r == nds[root].r) {
			nds[root].add(val);
			return;
		}
		int m = nds[root].mid();
		if (r <= m) {  // only left part
			add(l, r, val, lchild(root));
		} else if (l >= m) {  // only right part
			add(l, r, val, rchild(root));
		} else {
			add(l, m, val, lchild(root));
			add(m, r, val, rchild(root));
		}
		merge_flags(root);
	}

	item_type get_max(int l, int r, int root = 0) {
		if (l == nds[root].l && r == nds[root].r) {
			return nds[root].mx;
		}
		// 勿忘加上all_add lazy标记
		int m = nds[root].mid();
		if (r <= m) {  // only left part
			return get_max(l, r, lchild(root)) + nds[root].all_add;
		} else if (l >= m) {  // only right part
			return get_max(l, r, rchild(root)) + nds[root].all_add;
		} else {
			return max(get_max(l, m, lchild(root)),  // left
									get_max(m, r, rchild(root)))  // right
							+ nds[root].all_add;
		}
	}

	item_type get_min(int l, int r, int root = 0) {
		if (l == nds[root].l && r == nds[root].r) {
			return nds[root].mn;
		}
		// 勿忘加上all_add lazy标记
		int m = nds[root].mid();
		if (r <= m) {  // only left part
			return get_min(l, r, lchild(root)) + nds[root].all_add;
		} else if (l >= m) {  // only right part
			return get_min(l, r, rchild(root)) + nds[root].all_add;
		} else {
			return min(get_min(l, m, lchild(root)),  // left
									get_min(m, r, rchild(root)))  // right
							+ nds[root].all_add;
		}
	}

	sum_type get_sum(int l, int r, int root = 0) {
		if (l == nds[root].l && r == nds[root].r) {
			return nds[root].sm;
		}
		// 勿忘加上all_add lazy标记
		int m = nds[root].mid();
		ll lazy = nds[root].all_add * (sum_type)(r - l);
		if (r <= m) {  // only left part
			return get_sum(l, r, lchild(root)) + lazy;
		} else if (l >= m) {  // only right part
			return get_sum(l, r, rchild(root)) + lazy;
		} else {
			return get_sum(l, m, lchild(root))    // left
							+ get_sum(m, r, rchild(root))  // right
							+ lazy;
		}
	}

	void out() {
		cout << "n = " << n << "\n";
		for (int i = 0; i < n; ++i) cout << a[i] << " ";
		cout << "\n";
	}
};
};  // namespace lly

lly::segment_tree tr;

int main() {
	// 洛谷 P3372 【模板】线段树 1
	
	int n, m;
	//cin>>n>>m;
	scanf("%d%d",&n,&m);
	tr.init(n);
	tr.build();

	int o,x,y;
	ll k;
	for (int i = 0; i < m; ++i) {
		//cin>>o>>x>>y;
		scanf("%d%d%d",&o,&x,&y);
		--x;
		if (o == 1) { // [x,y)内都加上k
			scanf("%lld",&k);
			tr.add(x,y,k);
		} else { // 询问区间和
			auto sum = tr.get_sum(x,y);
			printf("%lld\n",sum);
		}
	}
	return 0;
}	
\end{lstlisting}

\subsection{区间集体加与乘+查询区间和线段树模板}
\begin{lstlisting}[language={c++}]
#include <bits/stdc++.h>
using namespace std;
typedef long long ll;

namespace lly {
// 所有下标符合C++风格
// 奢侈做法:使用全局变量开4倍数组。
// 考场抄代码所有item_type, sum_type 换成ll即可
// 支持区间和查询(%mod) 区间集体乘k, 区间集体加k
// kMaxItemSize是最大的原始数据数组的大小
// 区间加 下传所有路径上的mlt标记
// 区间乘 下传所有路径上的mlt标记和add标记
// 注意考场上抄代码敲完一个函数,别的函数可以复制然后粘贴替换,别少替换了。
struct segment_tree {
	// 最大的初始数组大小和响应的线段树节点数组最大大小。
	const static int kMaxItemSize = 100000; // 可手动更换
	const static int kMaxSegTreeSize = kMaxItemSize << 2;

	ll mod;

	int item_sz;
	int seg_sz;
	int &n = item_sz;
	int &stn = seg_sz;

	typedef ll item_type;
	typedef ll sum_type;

	struct nd {
		int l, r;
		sum_type sm;  // sum flag
		// lazy flags
		item_type all_add;  // lazy标记,表示区间内的数都要加上的数
		sum_type all_mlt;   // lazy标记,表示区间内的数都要乘以的数
		// 运算顺序,先乘后加,即先乘以all_mlt再加上all_add;先儿子运算,后父亲运算。
		// 因此区间×k操作,需要把all_add标记乘以k

		// other flags
		inline int mid() { return l + (r - l) / 2; }
		inline void set_basic(int l, int r) {
			this->l = l;
			this->r = r;
			all_add = 0;
			all_mlt = 1;
		}
		inline void add(item_type a, ll mod) {
			a %= mod;
			all_add = (all_add + a) % mod;
			sm = (sm + (sum_type)(r - l) * a) % mod;
		}
		inline void mlt(item_type m, ll mod) {
			m %= mod;
			all_add = (all_add * (sum_type)m) % mod;
			all_mlt = (all_mlt * m) % mod;
			sm = (sm * m) % mod;
		}
	};

	item_type a[kMaxItemSize];
	nd nds[kMaxSegTreeSize];

	void init() {  // 屏幕输入数组a版本,需要先确定n
		for (int i = 0; i < n; ++i) scanf("%lld", a + i);  // cin >> a[i];
		seg_sz = (2 << (int)(ceil(log2(item_sz)))) - 1;
	}

	void init(item_type src[], int cnt) {  // 内存数组输入数组a版本
		n = cnt;
		for (int i = 0; i < n; ++i) a[i] = src[i];
		seg_sz = (2 << (int)(ceil(log2(item_sz)))) - 1;
	}

	inline int parent(int x) { return (x - 1) >> 1; }
	inline int lchild(int x) { return (x << 1) | 1; }
	inline int rchild(int x) { return (x << 1) + 2; }

	inline void build() { build(0, n, 0); }

	inline void set_flags(int root, int i) {  // nds[root]用a[i]设置各类标志
		auto &p = nds[root];
		p.sm = a[i];
	}

	inline void merge_flags(int root) {
		auto &p = nds[root];
		auto &l = nds[lchild(root)];
		auto &r = nds[rchild(root)];
		p.sm = (l.sm + r.sm) % mod * (p.all_mlt) % mod +
						((sum_type)(p.r - p.l) * p.all_add) % mod;
		p.sm %= mod;
	}

	inline void down_mlt_flag(int root) {
		auto &p = nds[root];
		auto &l = nds[lchild(root)];
		auto &r = nds[rchild(root)];
		l.mlt(p.all_mlt, mod);
		r.mlt(p.all_mlt, mod);
		p.all_mlt = 1;
	}

	inline void down_flags(int root) {  // mlt and add
		auto &p = nds[root];
		auto &l = nds[lchild(root)];
		auto &r = nds[rchild(root)];
		l.mlt(p.all_mlt, mod);
		r.mlt(p.all_mlt, mod);
		p.all_mlt = 1;
		l.add(p.all_add, mod);
		r.add(p.all_add, mod);
		p.all_add = 0;
	}

	void build(int l, int r, int root) {
		nds[root].set_basic(l, r);
		if (l + 1 == r) {
			set_flags(root, l);
			return;
		}
		int m = l + (r - l) / 2;
		build(l, m, lchild(root));
		build(m, r, rchild(root));
		merge_flags(root);
	}

	// [l,r)区间的数都加上val
	void add(int l, int r, item_type val, int root = 0) {
		if (l == nds[root].l && r == nds[root].r) {
			nds[root].add(val, mod);
			return;
		}
		down_mlt_flag(root);
		int m = nds[root].mid();
		if (r <= m) {  // only left part
			add(l, r, val, lchild(root));
		} else if (l >= m) {  // only right part
			add(l, r, val, rchild(root));
		} else {
			add(l, m, val, lchild(root));
			add(m, r, val, rchild(root));
		}
		merge_flags(root);
	}

	// [l,r)区间的数都乘以val
	void mlt(int l, int r, item_type val, int root = 0) {
		if (l == nds[root].l && r == nds[root].r) {
			nds[root].mlt(val, mod);
			return;
		}
		down_flags(root);
		int m = nds[root].mid();
		if (r <= m) {  // only left part
			mlt(l, r, val, lchild(root));
		} else if (l >= m) {  // only right part
			mlt(l, r, val, rchild(root));
		} else {
			mlt(l, m, val, lchild(root));
			mlt(m, r, val, rchild(root));
		}
		merge_flags(root);
	}

	sum_type get_sum(int l, int r, int root = 0) {
		if (l == nds[root].l && r == nds[root].r) {
			return nds[root].sm;
		}
		// 勿忘加上all_add lazy标记
		int m = nds[root].mid();
		ll lazy = (nds[root].all_add * (sum_type)(r - l)) % mod;
		ll tmp;
		if (r <= m) {  // only left part
			tmp = get_sum(l, r, lchild(root));
		} else if (l >= m) {  // only right part
			tmp = get_sum(l, r, rchild(root));
		} else {
			tmp = get_sum(l, m, lchild(root))     // left
						+ get_sum(m, r, rchild(root));  // right
		}
		tmp %= mod;
		return (tmp * nds[root].all_mlt % mod + lazy) % mod;
	}

	void out() {
		cout << "n = " << n << "\n";
		for (int i = 0; i < n; ++i) cout << a[i] << " ";
		cout << "\n";
	}
};
};  // namespace lly

lly::segment_tree tr;

int main() {
	// 洛谷 P3373 【模板】线段树 2

	int n, m;
	ll mod;
	// cin>>n>>m;
	scanf("%d%d%lld", &n, &m, &mod);
	tr.n = n;
	tr.mod = mod;
	tr.init();
	tr.build();

	int o, x, y;
	ll k;
	for (int i = 0; i < m; ++i) {
		// cin>>o>>x>>y;
		scanf("%d%d%d", &o, &x, &y);
		--x;
		if (o == 2) {  // [x,y)内都加上k
			scanf("%lld", &k);
			tr.add(x, y, k);
		} else if (o == 3) {  // 询问区间和
			auto sum = tr.get_sum(x, y);
			printf("%lld\n", sum);
		} else {
			scanf("%lld", &k);
			tr.mlt(x, y, k);
		}
	}
	return 0;
}

\end{lstlisting}