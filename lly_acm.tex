%!TEX program = xelatex
\documentclass{article}
%\documentclass[utf8]{ctexart}
\usepackage[UTF8]{ctex}
\usepackage{fontspec}
\usepackage{listings}
\usepackage{xeCJK}

\usepackage{xparse, mathtools}
\usepackage{amsmath}
\usepackage{geometry}%页边距

%设置也的上下左右边距
\geometry{left=3.17cm,right=3.17cm,top=2.54cm,bottom=2.54cm}

\usepackage[dvipsnames, svgnames, x11names]{xcolor} % 定制颜色
\definecolor{mygreen}{rgb}{0,0.6,0}
\definecolor{mygray}{rgb}{0.5,0.5,0.5}
\definecolor{mymauve}{rgb}{0.58,0,0.82}

%行向量
\NewDocumentCommand{\Rowvec}{ O{,} m }
{
	\vector_main:nnnn { p } { & } { #1 } { #2 }
}
\NewDocumentCommand{\Colvec}{ O{,} m }
{
	\vector_main:nnnn { p } { \\ } { #1 } { #2 }
}

\setmainfont{Candara}[BoldFont={Microsoft JhengHei}, ItalicFont={Corbel}]
%\setCJKmainfont{STKaiti}[BoldFont={FandolHei-Bold}, ItalicFont={STXingkai}]
\setCJKmainfont{FZYBKSJW.TTF}[BoldFont={FandolHei-Bold.otf}, ItalicFont={STXingkai.ttf}] %正文字体
%[BoldFont={FandolHei-Bold}, ItalicFont={STXingkai}]1
\setmonofont{Consolas} %inconsolata
\newcommand{\shusong}{\CJKfontspec{FZShuSong-Z01S}} %方正书宋
\newcommand{\xingkai}{\CJKfontspec{STXingkai}}%华文行楷
\usepackage{indentfirst}%设置缩进
\setlength{\parindent}{2em} %也可在正文中非全局设置



\lstset{ %
	backgroundcolor=\color{white},      % choose the background color
	basicstyle=\small\ttfamily,  % size of fonts used for the code
	numbers=none,                                        % 在左侧显示行号
  numberstyle=\small\color{Dandelion},                       % 设定行号格式
	columns=fixed,
	tabsize=2,
	breaklines=true,               % automatic line breaking only at whitespace
	captionpos=b,                  % sets the caption-position to bottom
	commentstyle=\color{mygreen},  % comment style
	escapeinside={\%*}{*)},        % if you want to add LaTeX within your code
	keywordstyle=\color{blue},     % keyword style
	stringstyle=\color{mymauve}\ttfamily,  % string literal style
	frame=double,
	rulesepcolor=\color{red!20!green!20!blue!20},
	identifierstyle=\color{VioletRed},
	morekeywords={ll},
	language=c++,
}



\begin{document}
	% 目录区
	\tableofcontents
	% 目录区
	
	\section{动态规划}
		\subsection{背包问题}
			\subsubsection{0-1背包}
				\par n个物体,每个物体的费用为$cost_i$,价值为$val_i$.最大允许费用为$max\_cost$,问能挑选出来的最大价值是什么。
				\begin{lstlisting}[language={c++}]
for (int i = 1; i <= max_cost; i++)
	for (int l = max_cost - i; l >= 0; l--)
		f[l + i] = max(f[l] + cost[i], f[l + i]);		
				\end{lstlisting}
	\section{拓展欧几里得算法}
	\par 欧几里得算法直接使用g++中的<algorithm>库中\_\_gcd()函数即可。
	\par $(a,b)=(b,a \bmod b)$.
	\par 拓展欧几里得算法用于求出不定方程$ax+by=(a,b)$的一个特解$x_0,y_0$, 顺带求出$(a,b)$,{\bfseries 通解$x=x_0+\frac{b}{(a,b)}t,\,y=y_0-\frac{a}{(a,b)}t \; (t \in Z)$}.
	\subsection{解不定方程}
		\par 不定方程$ax+by=c$有解等价于$(a,b) \mid c$.据此判断是否有解,若有解,则原不定方程的解是(且只能是)欧几里得算法所解不定方程解的$\frac{c}{(a,b)}$倍。
	\subsection{求解模线性方程(线性同余方程)}
		$ax \equiv c\pmod{m} \Longleftrightarrow ax+my=c$.
	\subsection{求乘法逆元}
		\par $ab \equiv 1 \pmod{m}$, 则a关于模m的乘法逆元是b,b关于模m的乘法逆元是a。或者说$Z_m$群中a和b互为乘法逆元。
		\par 用乘法逆元有$\frac{A}{b} \equiv A \times b^{-1} \pmod{c}$.当左边的式子A是很大的数,而b是小规模数,且除出来的数一定是整数的时候,可以用右式边算边模。
		\par \textbf{求解} $ax \equiv 1 \pmod{m} \Longleftrightarrow ax+my= 1$.解出的x即为解,只是注意需要用通解公式将$x$调整到$Z_m$范围内。
	\subsection{线性同余方程组}
		\par 方程组$a_ix \equiv c_i \pmod{m_i} \quad (i=1, 2, 3, \ldots, n)$
		\par 对每一个现行同余方程,若无解,则方程组无解;否则,可以解得$x=k_it_i+x_i, \; t_i \in Z$.而$x_i \in \left[0,k_i\right), k_i = \frac{m_i}{(a_i,m_i)} \leq m_i$.不妨增加$x_0=0,k_0=1$,即增加式子$x=t_0+0$.
		\par 现在考虑同时满足$x=k_1t_1+x_1$与$x=k_2t_2+x_2$两个约束的x能否合并成一个依旧如此形式的一个表达式,即$x=k_0t+x_0$.
		\par 联立两个方程,易得$k_1t_1-k_2t_2=x_2-x_1$,将其视作关于$t_1,t_2$的不定方程。若这个方程无解,说明同时满足两个条件的x不存在;否则,每确定一个$t_1$可以代入$x=k_1t_1+x_1$确定一个$x$. 如果我们只是解出$t_1$的话,不妨换成$k_1t_1+k_2t_2=x_2-x_1$,$t_1$的每一个解是不变的。
		\par 假设$k_1t_1+k_2t_2=x_2-x_1$有解,并且解为$t_1=t_{1_0}+\frac{k_2}{(k_1,k_2)}t \quad t_{1_0} \in \left[ 0,\frac{k_2}{(k_1,k_2)} \right), \; t \in Z$.
		\par 代入$x=k_1t_1+x_1$得$x=k_1(t_{1_0}+\frac{k_2}{(k_1,k_2)}t)+x_1=\frac{k_1k_2}{(k_1,k_2)}t+(k_1t_{1_0}+x_1)$.而$k_1t_{1_0}+x_1 < \frac{k_1k_2}{(k_1,k_2)} \Longleftrightarrow t_{1_0}+\frac{x_1}{k_1} < \frac{k_2}{(k_1,k_2)}$.而$t_{1_0} < \frac{k_2}{(k_1,k_2)}$.注意到$t_{1_0},\frac{k_2}{(k_1,k_2)}$是整数,故$t_{1_0} \leq \frac{k_2}{(k_1,k_2)}-1$.又$\frac{x_1}{k_1}\, <\, 1$.这两个不等式相加即可得到$$t_{1_0}+\frac{x_1}{k_1} < \frac{k_2}{(k_1,k_2)}$$。即符合前面定义的形式:$$(k_1t_{1_0}+x_1) \in \left[0, \frac{k_1k_2}{(k_1,k_2)}\right)$$.
		\par 合并为$x=kt+x_0.$的形式有:\begin{align*}
			k &=\frac{k_1,k_2}{(k_1,k_2)}=[k_1,k_2] \\
			x_0 &=k_1t_{1_0}+x_1
		\end{align*}
		\par 对于写程序,由于我们引入了$x=x_0=0,k=k_0=1$.可以每次$x_0,k_0$与$x_i,k_i$合并成$x_0,k_0$.故程序迭代是$x+=kt,k=[k,k_i]$,其中$t_0$是$k_0t_0+k_it_i=x_i-x_0$的不定方程的最小非负整数解。
		\subsubsection{程序设计方面爆long long的问题及应对策略}
		\par {\bfseries 由于k的迭代是不断求最小公倍数,而特解x始终是小于k的,因此k和x可能会增长的很快导致爆long long.尤其是k很容易爆掉。}
		\par 如何解决爆炸的问题?或许可以用{\textbf \_\_int128}。如果k和x还是都爆掉了,那么没法子,只能设法自己实现k和x的存储,注意到解不定方程需要求$(k_0,k_i)$与$\frac{k_i}{(k_0,k_i)}$,所以要大数加减,大数乘除模普通数的实现。估计够呛。
		\begin{lstlisting}[language={c++}]
// poj 2891 然而poj不支持__int128和C++11
#include <bits/stdc++.h>
typedef __int128 ll;

// 求解不定方程ax+by=(a,b)的一组特解并返回a,b最大公约数
// x,y存储返回的一组特解。易懂version
ll ex_gcd1(ll a, ll b, ll &x, ll &y) {
	if (b) {
		auto d = ex_gcd1(b, a%b, x,  y);
		auto x_bac = x;
		x = y; // x设为后
		y = x_bac - a/b * y; // y设为前-a/b*后
		return d;
	} else {
		x = 1; y = 0;
		return a;
	}
}

// 求解不定方程ax+by=(a,b)的一组特解并返回a,b最大公约数
// x,y存储返回的一组特解。
ll ex_gcd(ll a, ll b, ll &x, ll &y) {
	if (b) {
		auto d = ex_gcd(b, a%b, y, x); // 注意x和y位置互换了。
		// x是后,无需赋值,y是 前-a/b*后 即 y -= a/b*x
		y -= a/b*x;
		return d;
	} else {
		x = 1; y = 0;
		return a;
	}
}

// 求解不定方程ax+by=c.
// 返回值表示是否有解
// d存储是(a,b)
// 当有解的情况下
// x,y存储一组特解,并且确保x是最小的非负整数。
// 通解是X=x+(b/d)*t,Y=y-(a/d)*t t是整数。
bool binary_linear_indefinite_equation(ll a, ll b, ll c, ll &x, ll &y, ll &d) {
	d = ex_gcd(a, b, x, y); // solve: ax+by=(a,b)
	if (c%d) return false;
	x *= c/d;
	y *= c/d;
	auto k = b/d;
	x = (x%k+k)%k; // 调为最小非负整数
	y = (d-a*x)/b;
	return true;
}

// 线性同余方程 Linear congruence equation
// ax = c (mod m) <===> ax+my=c
// x存储最小非负解,通解X=x+kt t为整数
// 有解的情况下,最小非负解x肯定在[0,m)范围内
bool linear_congruence_equation(ll a, ll c, ll m, ll &x, ll &k) {
	ll y, d;
	auto ans = binary_linear_indefinite_equation(a, m, c, x, y, d);
	k = m/d;
	return ans;
}

// 求a在Zm<+,*>中的乘法逆元x
// 返回逆元是否存在,x存储逆元
// ax = 1 (mod m)
bool multiplicative_inverse(ll a, ll m, ll &x) {
	ll k;
	return linear_congruence_equation(a, 1, m, x, k);
	// assert(k == m);
}

// 线性同余方程组 Linear congruence equations
// a_ix = c_i (mod m_i) 共n个
// 可能存在的问题,由于迭代过程中k一直在求最小公倍数,所以可能会爆long long,这个,最佳的方法是直接暴力把ll的定义改为__int128
// 但是要注意__int128的输入输出
// 如果还是爆,我没法子了
bool linear_congruence_equations(int n, ll a[], ll c[], ll m[], ll &x, ll &k) {
	ll x_i, k_i, t, t_i, d;
	x = 0; k = 1;
	for (int i = 0; i < n; ++i) {
		if (!linear_congruence_equation(a[i], c[i], m[i], x_i, k_i))
			return false; 
		// kt+x
		// k_it_i+x_i
		if (!binary_linear_indefinite_equation(
			k, k_i, x_i-x, t, t_i, d))
			return false;
		x += k*t;
		k *= k_i/d;
	}
	return true;
}

inline ll read()
{
	ll x = 0;
	bool f = 0;
	char ch = getchar();
	while (ch < '0' || '9' < ch)
		f |= ch == '-', ch = getchar();
	while ('0' <= ch && ch <= '9')
		x = x * 10 + ch - '0', ch = getchar();
	return f ? -x : x;
}

void write(ll a)
{
	if (a < 0)
	{
		putchar('-');
		a = -a;
	}
	if (a >= 10)
	{
		write(a / 10);
	}
	putchar(a % 10 + '0');
}

const int kMaxN = 10000;
ll a[kMaxN]; // ax=c (mod c)
ll m[kMaxN];
ll c[kMaxN]; 
void solve(int n) {
	for (int i = 0; i < n; ++i) {
		a[i] = 1;
		m[i] = read();
		c[i] = read();
	}
	ll x,k;
	auto ans = linear_congruence_equations(n,a,c,m,x,k);
	if (ans) {
		write(x);
		putchar('\n');
	} else
		puts("-1");
}

int main()
{
	int n;
	while (scanf("%d",&n) != EOF) {
		solve(n);
	}
}
		\end{lstlisting}
\section{快速乘幂及矩阵快速幂}
    \subsection{快速模模乘幂}
        \par 时间复杂度: 快速乘、普通快速幂$O(\log_2{n})$,使用快速乘的快速幂$O(\log_2{n} \times \log_2{max_val})=O(\log_2{n} \times \log_2{mod})$
        \begin{lstlisting}[language={c++}]
struct mod_sys{
    typedef long long ll;
    ll mod;
    // mod_sys类初始化设置模数
    inline void set_mod(ll mod0) {mod = mod0;}
    // 返回a在[0,mod)内标准等价的数,即数学意义上的a%mod
    inline ll to_std(ll a) {return (a%mod+mod)%mod;}
    // 计算数学意义上的a*n%mod
    ll mlt(ll a, ll n) {
        a = to_std(a); n = to_std(n);
        if (0 == a || 0 == n) return 0;
        // 始终维持要求的数可以表示为n(a)+t
        ll t = 0;
        while (n > 1) {
            if (n&1) t = (t+a)%mod;
            n >>= 1; a = (a<<1)%mod;
        }
        return (a+t)%mod; // now n = 1
    }
    // 计算数学意义上的a^n%mod 输入应当a,n>=0
    ll pow(ll a, ll n)
    {   
        if (n == 0) return 1%mod;
        a = to_std(a);
        // 始终维持要求的数可以表示为(a)^n*t
        ll t = 1;
        while (n > 1)
        {
            if (n&1) t = t*a%mod;
            n >>= 1; a = a*a%mod;
        }
        return a*t%mod; // now n = 1
    }
    // 计算数学意义上的a^n%mod 输入应当a,n>=0
    // 此版本使用quick_mlt防止相乘爆ll
    ll pow_v2(ll a, ll n)
    {   
        if (n == 0) return 1%mod;
        a = to_std(a);
        // 始终维持要求的数可以表示为(a)^n*t
        ll t = 1;
        while (n > 1)
        {
            if (n&1) t = mlt(t,a);
            n >>= 1; a = mlt(a,a);
        }
        return mlt(t,a); // now n = 1
    }
};
        \end{lstlisting}
    \subsection{矩阵乘法}
        时间复杂度:$n \times m$与$m \times r$的矩阵相乘,复杂度$O(nmr)$。
        \begin{lstlisting}[language={c++}]
        \end{lstlisting}
    \subsection{矩阵快速幂}
        计算$A^n$.矩阵乘法的次数$O(\log_2{n})$,总复杂度$|A|^3\log_2{n}$.
        \begin{lstlisting}[language={c++}]
        \end{lstlisting}
\section{Miller-Rabin素数检测算法}
其基于以下两个定理。
\begin{enumerate}
    \item Fermat小定理 若n是素数,则$\forall a(a \not\equiv 0 \pmod{n})$,有$a^{n-1} \equiv 1 \pmod{n}$.
    \item 二次探测定理 若n是素数,则$x^2 \equiv 1 \pmod{n}$只有平凡根$x=\pm1$,即$x=1,x=n-1$.
\end{enumerate}
\subsection{具体算法}
假设$n$是奇数,令$n=m \times 2^q (q \geq 1)$,其中$m$是奇数.
对于序列$a^m \bmod n, a^{2m} \bmod n, a^{4m} \bmod n,\ldots,a^{2^q \times m} \bmod n$.
最后一项就是费马小定理中的$a^{n-1}$, 并且每一项都是前一项的平方。
我们一项一项往后计算。
\begin{itemize}
    \item 若当前项为1,后面每一项显然都是1。而根据二次探测定理,n是素数必须前面一项是1或n-1.如果不符合,断言不是素数;符合,断言是素数。
    \item 若当前项不是1,暂时不断言,接着往后算。除非当前是最后一项了,那么断言不是素数。
\end{itemize}
\textbf{当然,如果第一项是1,由于不存在二次探测的方程,所以不检验前面一项(或者认为前面一项符合条件)。}
\subsection{Code}
使用了快速幂模和快速幂加模板mod\_sys。下面代码只是miller-rabin核心代码。
\begin{lstlisting}[language={c++}]
// 如果只是int范围内,可以将pow_v2改为pow,mlt改为普通乘法
bool miller_rabin(ll a, ll n, ll q, ll m, mod_sys& mod) {
    a = mod.pow_v2(a, m);
    bool is_ordinary = true;
    for (int i = 0; i < q; ++i) {
        if (a == 1) {
            return is_ordinary;
        } else {
            is_ordinary = (a == n-1);
            a = mod.mlt(a,a);
        }
    }
    return (a==1)&&(is_ordinary); // 最后一项
}

// 使用miller_rabin检测是否是素数
const int kCheckCnt = 8;
// 为了随机数
random_device rd;
mt19937_64 gen(rd());
bool miller_rabin(ll n) {
    if (n == 2) return true;
    if ((n <= 2) || (n&1^1)) return false;
    // 2^q×m表示原本输入的n-1
    ll m = n, q = 0;
    do { m >>= 1; ++q; } while(m&1^1);
    // 随机数生成,[1,n-1] 均匀分布
    uniform_int_distribution<> dis(1, n-1);
    mod_sys mod;
    mod.set_mod(n);
    for (int i = 0; i < kCheckCnt; ++i)
        if (!miller_rabin(dis(gen), n, q, m, mod))
            return false;
    return true;
}
\end{lstlisting}
\section{lucas定理}
    \par $$\tbinom{n}{m} \bmod p = \tbinom{\lfloor \frac{n}{p} \rfloor}{\lfloor \frac{m}{p} \rfloor} \tbinom{n \bmod p}{m \bmod p} \bmod p=\tbinom{n/p}{m/p}\tbinom{n\%p}{m\%p} \bmod p$$  
    \par 先预先求出$i! \;(i \in \left[0,p\right))$. 并利用费马小定理和快速幂乘求出每一个$i!$的逆元$(i!)^{-1}$。求$\tbinom{n}{m} \bmod p$,当$m=0$直接就是$1$.若$n,m$都在$p$范围内,则直接转化为$n! \times (m!)^{-1} \times [(n-m)!]^{-1}$.否则就是\textit{lucas定理}缩小规模。
    \par {\bfseries 对一个固定的p,预处理求阶乘及快速模幂求其逆元,时间复杂度$O(p\log_2{p})$。空间复杂度$O(p)$。预处理之后,单次求$\tbinom{n}{m} \bmod p$复杂度$O(\log_{p}{m})$} 
    \begin{lstlisting}[language={c++}]
void prepare(ll p, vector<ll>&fac, vector<ll>&inv_fac) {
    fac.resize(p); inv_fac.resize(p);
    mod_sys mod;
    mod.set_mod(p);
    fac[0] = 1;
    inv_fac[0] = 1;
    for (int i = 1; i < p; ++i) {
        fac[i] = (fac[i-1]*i)%p;
        inv_fac[i] = mod.pow(fac[i], p-2); // 既然能枚举一遍,p*p不应该爆ll
    }
}

// 输入预设0=<n,m<p
inline ll combination(ll n, ll m, ll p, vector<ll>&fac, vector<ll>&inv_fac) {
    if (n < m) return 0;
    return fac[n]*inv_fac[m]%p*inv_fac[n-m]%p;
}

ll lucas(ll n, ll m, ll p, vector<ll>&fac, vector<ll>&inv_fac) {
    if (n < m) return 0;
    ll ans = 1;
    while(true) {
        if (m == 0) return ans;
        if (n < p && m < p) return ans*combination(n,m,p,fac,inv_fac)%p;
        ans = ans * combination(n%p,m%p,p,fac,inv_fac)%p;
        n/=p; m/=p;
    }
}
    \end{lstlisting}
\section{欧拉函数}
	\par $\phi(n)$表示1到n中与n互素的个数.
	\par 公式 $\phi(n)=n\prod\limits_{p \mid n \& p \in P } 1-\frac{1}{p}$
	\par $\phi(1..n)$总体求解可用线性筛$O(n)$求出.
	\par 单个$\phi(n)$可用分解质因数法直接用公式$O(\sqrt{n}$求出。
	
	%code block
\begin{lstlisting}[language={c++}]
// 时间复杂度sqrt(n)求phi(n) n最大1e12-1e14的级别
// more beautiful version, but slower (just a little bit)
ll phi(ll n) {
	ll a = n;
	for (ll p = 2; p * p <= n; ++p)
		if (!(n % p)) {
			do
				n /= p;
			while (!(n % p));
			a = a / p * (p - 1);
		}
	if (n > 1) a = a / n * (n - 1);  // the rest n is a prime
	return a;
}

// 计算1--n的所有phi(i) 线性时空复杂度,n应该最大是1e7级别的
void get_all_phi(int n, vector<int>& phi) {
	phi.resize(n + 1);
	vector<bool> is_prime(n + 1, true);
	vector<int> prime;
	is_prime[1] = is_prime[0] = false;
	phi[1] = 1;
	for (int i = 2; i <= n; ++i) {
		if (is_prime[i]) {
			prime.push_back(i);
			phi[i] = i - 1;
		}
		for (auto p : prime) {
			if (i * p > n) break;
			is_prime[i * p] = false;
			if (i % p) {
				// i不具有素因子p,i*p对于素因子p来讲次数=1。贡献是(p-1)
				phi[i * p] = phi[i] * (p - 1);
			} else {
				phi[i * p] = phi[i] * p;  // i具有素因子p,则i*p对于欧拉函数值来讲乘以p
				break;  // 保证每个数只被最小素因子访问到。
			}
		}
	}
}
\end{lstlisting}
\section{线性筛}
	\par 时空复杂度$O(n).$

\begin{lstlisting}[language={c++}]
void linear_sieve(int n, vector<int>& f) {
	f.resize(n + 1);
	vector<bool> is_prime(n + 1, true);
	vector<int> prime;
	is_prime[1] = is_prime[0] = false;
	f[1] = 1;
	for (int i = 2; i <= n; ++i) {
		if (is_prime[i]) {
			prime.push_back(i);
			// code here for i 当i为素数
		}
		for (auto p : prime) {
			if (i * p > n) break;
			is_prime[i * p] = false;
			if (i % p) {
				// code here for (i * p), 当(i * p)关于素因子p的次数大于等于2
			} else {
				// code here for (i * p), 当(i * p)关于素因子p的次数仅仅为1
				break;  // 保证每个数只被最小素因子访问到。保证线性。
			}
		}
	}
}
\end{lstlisting}

\section{min 25筛}
\par {\bfseries min 25筛}即{\bfseries 基于质因数分解的亚线性函数前缀和求法},可以在$O(\frac{n^{\frac{3}{4}}}{\log{n}})$的时间内求积性函数$f(x)$的前缀和。要求$f(p)$是一个关于$p$的简单多项式,$f(p^c)$可以快速计算。


\section{莫比乌斯反演}
		\par $F$是已知函数,$f$是未知函数。$\mu(x)$是固定函数,即\textbf{莫比乌斯函数}。
		\subsection{莫比乌斯函数$\mu(x)$}
			\begin{itemize}
				\item $\mu(1)=1$
				\item x为不同的质数的乘积。若质数个数为奇数,则$\mu(x)=-1$; 偶数个$\mu(x)=1$.
				\item 剩下的情况,即x的某个素因子的次数大于等于2. $\mu(x)=0$.
			\end{itemize}
			\par 莫比乌斯函是积性函数,$10^7$规模的数据,故可用\textbf{线性筛思想}解决,否则需要使用杜教筛。
			%code block
			\begin{lstlisting}[language={c++}]
// 计算所有的\mu(x) x in [1..n]. 线性复杂度。
void get_all_mu(int n, vector<int>&mu) {
  mu.resize(n+1);
  vector<bool>is_prime(n+1, true);
  vector<int>prime;
  is_prime[1] = is_prime[0] = false; mu[1] = 1;
  for (int i = 2; i <= n; ++i) {
    if (is_prime[i]) {
      prime.push_back(i);
      mu[i] = -1;
    }
    for (auto p : prime) {
      if (i*p > n) break;
      is_prime[i*p] = false;
      if (i%p) {
        mu[i*p] = -mu[i]; // i不具有素因子p,具有积性
      } else {
        mu[i*p] = 0; // i具有素因子p,则i*p具有素因子的平方的因子
        break; // 保证每个数只被最小素因子访问到。
      }
    }
  }
}
			\end{lstlisting}

		\subsection{整除分块}
		基本形式$\sum\limits_{i=1}^{n} \lfloor \frac{n}{i} \rfloor$.假设n的因数从小到大保存在d[1..k]中。
		\begin{equation}
			\mathbf{d} = 
			\left(
				\begin{array}{ccc}
					d_1, d_2,  d_3,  \ldots,  d_k
				\end{array}	
			\right)
		\end{equation}
		显然$i \in (d_j,d_{j+1}]$时$\lfloor \frac{n}{i} \rfloor$结果相同,假设结果为$k$, 则$d_{j+1}=\lfloor \frac{n}{k} \rfloor$。
		因此可以一段一段区间的跳跃求和,将复杂度降到$\sqrt{n}$.
		\par 大多数时候不是基本形式,而是基本形式乘以一个函数求和。这个时候,可以对这个函数求前缀和,然后用整除分块的代码做一下变通即可。
		
		%code block
		\begin{lstlisting}[language={c++}]
for(int l=1,r;l<=n;l=r+1) {
  r=n/(n/l);
  ans+=(r-l+1)*(n/l)*(sum[r]-sum[l-1]); // sum是乘以函数的前缀和 
}
		\end{lstlisting}

		\subsection{莫比乌斯反演定理}
			\subsubsection{因数形式}
			\par d取遍n的所有因数。$F(n)=\sum\limits_{d \mid n}f(d)$,反演求出未知的$f$的表达式为$f(n)=\sum\limits_{d \mid n}\mu(d)F(\frac{n}{d})=\sum\limits_{d \mid n}F(d)\mu(\frac{n}{d})$
			\subsubsection{倍数形式}
			\par d取遍n的所有倍数。$F(n)=\sum\limits_{n \mid d}{f(d)}$,反演求出未知的$f$的表达式为$f(n)=\sum\limits_{n \mid d}{\mu(\frac{d}{n})F(d)}$.
		\subsection{做题思路或者技巧}
			\begin{enumerate}
				\item 构造能够直接写出表达式的已知函数F,然后尝试因数或者倍数形式的小f.
				\item 利用反演定理求出f的表达式。
				\item 将ans用小f的形式表达出来。可能还是一个关于f的$\Sigma$求和式。
				\item 代入f的反演结果,根据数学尝试变换枚举变量的顺序。
				\item 与gcd有关的莫比乌斯反演。一般我们都是套路的去设$f(d)$\mbox{为}$gcd(i,j)=d$\mbox{的个数},$F(n)$\mbox{为}$gcd(i,j)=d\mbox{的倍数}$的个数。
				\item 注意最后的式子是不是包含一个整除分块的部分。
			\end{enumerate}

			%code block
			\begin{lstlisting}[language={c++}]
			\end{lstlisting}

	\section{线段树}
	\par 采用左闭右开区间模式。数组元素、线段树节点下标通通从0开始记。$p=(ch-1)/2, l=2p+1, r=2p+2.$
	\par 元素个数为$item\_sz$, 线段树节点数组大小$seg\_nd\_sz=2^{{\lceil \log_2{item\_sz}  \rceil} + 1}-1 < 2^{\log_2{item\_sz} + 1 + 1} = 4 \times item\_sz$. 放缩操作是向上取整的结果肯定小于直接加1的结果,并且还没有减去最后需要减去的1.\textit{所以,简单奢侈的方式是直接开4倍。}
	\par lazy标记,就是记录区间中元素共有的操作。节点的最值、和等标记是不考虑节点自身的lazy标记的结果,但是这些标记由子节点求解的时候需要考虑子节点的lazy标记;但是对于最值、和等的询问,返回的值是考虑了lazy标记的结果。
	\par 即\textit{ mx,mn,sum标记的值和get\_min,get\_mn,get\_sum的返回的值不一定相同}。
	\subsection{区间集体加+查询区间和最大最小值模板}

\begin{lstlisting}[language={c++}]
#include <bits/stdc++.h>
using namespace std;
typedef long long ll;

namespace lly {
// 所有下标符合C++风格
// 奢侈做法:使用全局变量开4倍数组。
// 最小值,最大值,区间和,区间加相同数模板
// 考场抄代码所有item_type, sum_type 换成ll即可
struct segment_tree {
	// 最大的初始数组大小和响应的线段树节点数组最大大小。
	const static int kMaxItemSize = 100000;
	const static int kMaxSegTreeSize = kMaxItemSize << 2;

	int item_sz;
	int seg_sz;
	int &n = item_sz;
	int &stn = seg_sz;

	typedef ll item_type;
	typedef ll sum_type;

	struct nd {
		int l, r;
		item_type mx;  // max
		item_type mn;  // min
		sum_type sm;   // sum flag
		// lazy flags
		item_type all_add;  // lazy标记,表示区间内的数都要加上的数
		// other flags
		inline void add(item_type a) {
			all_add += a;
			mx += a;
			mn += a;
			sm += (sum_type)(r-l)*a;
		}
		inline int mid() { return l + (r - l) / 2; }
	};

	item_type a[kMaxItemSize];
	nd nds[kMaxSegTreeSize];

	void init(int cnt) {  // 屏幕输入数组a版本
		n = cnt;
		for (int i = 0; i < n; ++i) scanf("%lld", a+i);//cin >> a[i];
		seg_sz = (2 << (int)(ceil( log2(item_sz) ))) - 1;
	}

	void init(item_type src[], int cnt) {  // 内存数组输入数组a版本
		n = cnt;
		for (int i = 0; i < n; ++i) a[i] = src[i];
		seg_sz = (2 << (int)(ceil( log2(item_sz) ))) - 1;
	}

	inline int parent(int x) { return (x - 1) >> 1; }
	inline int lchild(int x) { return (x << 1) | 1; }
	inline int rchild(int x) { return (x << 1) + 2; }

	inline void build() { build(0, n, 0); }

	inline void set_flags(int root, int i) {  // nds[root]用a[i]设置各类标志
		auto &p = nds[root];
		p.l = i;
		p.r = i + 1;
		p.mx = a[i];
		p.mn = a[i];
		p.sm = a[i];
	}

	inline void merge_flags(int root) {
		auto &p = nds[root];
		auto &l = nds[lchild(root)];
		auto &r = nds[rchild(root)];
		p.l = l.l;
		p.r = r.r;
		p.mx = max(l.mx, r.mx);
		p.mn = min(l.mn, r.mn);
		p.sm = l.sm + r.sm + p.all_add * (sum_type)(p.r - p.l);
	}

	void build(int l, int r, int root) {
		nds[root].all_add = 0;
		if (l + 1 == r) {
			set_flags(root, l);
			return;
		}
		int m = l + (r - l) / 2;
		build(l, m, lchild(root));
		build(m, r, rchild(root));
		merge_flags(root);
	}

	// [l,r)区间的数都加上val
	void add(int l, int r, item_type val, int root = 0) {
		if (l == nds[root].l && r == nds[root].r) {
			nds[root].add(val);
			return;
		}
		int m = nds[root].mid();
		if (r <= m) {  // only left part
			add(l, r, val, lchild(root));
		} else if (l >= m) {  // only right part
			add(l, r, val, rchild(root));
		} else {
			add(l, m, val, lchild(root));
			add(m, r, val, rchild(root));
		}
		merge_flags(root);
	}

	item_type get_max(int l, int r, int root = 0) {
		if (l == nds[root].l && r == nds[root].r) {
			return nds[root].mx;
		}
		// 勿忘加上all_add lazy标记
		int m = nds[root].mid();
		if (r <= m) {  // only left part
			return get_max(l, r, lchild(root)) + nds[root].all_add;
		} else if (l >= m) {  // only right part
			return get_max(l, r, rchild(root)) + nds[root].all_add;
		} else {
			return max(get_max(l, m, lchild(root)),  // left
									get_max(m, r, rchild(root)))  // right
							+ nds[root].all_add;
		}
	}

	item_type get_min(int l, int r, int root = 0) {
		if (l == nds[root].l && r == nds[root].r) {
			return nds[root].mn;
		}
		// 勿忘加上all_add lazy标记
		int m = nds[root].mid();
		if (r <= m) {  // only left part
			return get_min(l, r, lchild(root)) + nds[root].all_add;
		} else if (l >= m) {  // only right part
			return get_min(l, r, rchild(root)) + nds[root].all_add;
		} else {
			return min(get_min(l, m, lchild(root)),  // left
									get_min(m, r, rchild(root)))  // right
							+ nds[root].all_add;
		}
	}

	sum_type get_sum(int l, int r, int root = 0) {
		if (l == nds[root].l && r == nds[root].r) {
			return nds[root].sm;
		}
		// 勿忘加上all_add lazy标记
		int m = nds[root].mid();
		ll lazy = nds[root].all_add * (sum_type)(r - l);
		if (r <= m) {  // only left part
			return get_sum(l, r, lchild(root)) + lazy;
		} else if (l >= m) {  // only right part
			return get_sum(l, r, rchild(root)) + lazy;
		} else {
			return get_sum(l, m, lchild(root))    // left
							+ get_sum(m, r, rchild(root))  // right
							+ lazy;
		}
	}

	void out() {
		cout << "n = " << n << "\n";
		for (int i = 0; i < n; ++i) cout << a[i] << " ";
		cout << "\n";
	}
};
};  // namespace lly

lly::segment_tree tr;

int main() {
	// 洛谷 P3372 【模板】线段树 1
	
	int n, m;
	//cin>>n>>m;
	scanf("%d%d",&n,&m);
	tr.init(n);
	tr.build();

	int o,x,y;
	ll k;
	for (int i = 0; i < m; ++i) {
		//cin>>o>>x>>y;
		scanf("%d%d%d",&o,&x,&y);
		--x;
		if (o == 1) { // [x,y)内都加上k
			scanf("%lld",&k);
			tr.add(x,y,k);
		} else { // 询问区间和
			auto sum = tr.get_sum(x,y);
			printf("%lld\n",sum);
		}
	}
	return 0;
}	
\end{lstlisting}

\subsection{区间集体加与乘+查询区间和线段树模板}
\begin{lstlisting}[language={c++}]
#include <bits/stdc++.h>
using namespace std;
typedef long long ll;

namespace lly {
// 所有下标符合C++风格
// 奢侈做法:使用全局变量开4倍数组。
// 考场抄代码所有item_type, sum_type 换成ll即可
// 支持区间和查询(%mod) 区间集体乘k, 区间集体加k
// kMaxItemSize是最大的原始数据数组的大小
// 区间加 下传所有路径上的mlt标记
// 区间乘 下传所有路径上的mlt标记和add标记
// 注意考场上抄代码敲完一个函数,别的函数可以复制然后粘贴替换,别少替换了。
struct segment_tree {
	// 最大的初始数组大小和响应的线段树节点数组最大大小。
	const static int kMaxItemSize = 100000; // 可手动更换
	const static int kMaxSegTreeSize = kMaxItemSize << 2;

	ll mod;

	int item_sz;
	int seg_sz;
	int &n = item_sz;
	int &stn = seg_sz;

	typedef ll item_type;
	typedef ll sum_type;

	struct nd {
		int l, r;
		sum_type sm;  // sum flag
		// lazy flags
		item_type all_add;  // lazy标记,表示区间内的数都要加上的数
		sum_type all_mlt;   // lazy标记,表示区间内的数都要乘以的数
		// 运算顺序,先乘后加,即先乘以all_mlt再加上all_add;先儿子运算,后父亲运算。
		// 因此区间×k操作,需要把all_add标记乘以k

		// other flags
		inline int mid() { return l + (r - l) / 2; }
		inline void set_basic(int l, int r) {
			this->l = l;
			this->r = r;
			all_add = 0;
			all_mlt = 1;
		}
		inline void add(item_type a, ll mod) {
			a %= mod;
			all_add = (all_add + a) % mod;
			sm = (sm + (sum_type)(r - l) * a) % mod;
		}
		inline void mlt(item_type m, ll mod) {
			m %= mod;
			all_add = (all_add * (sum_type)m) % mod;
			all_mlt = (all_mlt * m) % mod;
			sm = (sm * m) % mod;
		}
	};

	item_type a[kMaxItemSize];
	nd nds[kMaxSegTreeSize];

	void init() {  // 屏幕输入数组a版本,需要先确定n
		for (int i = 0; i < n; ++i) scanf("%lld", a + i);  // cin >> a[i];
		seg_sz = (2 << (int)(ceil(log2(item_sz)))) - 1;
	}

	void init(item_type src[], int cnt) {  // 内存数组输入数组a版本
		n = cnt;
		for (int i = 0; i < n; ++i) a[i] = src[i];
		seg_sz = (2 << (int)(ceil(log2(item_sz)))) - 1;
	}

	inline int parent(int x) { return (x - 1) >> 1; }
	inline int lchild(int x) { return (x << 1) | 1; }
	inline int rchild(int x) { return (x << 1) + 2; }

	inline void build() { build(0, n, 0); }

	inline void set_flags(int root, int i) {  // nds[root]用a[i]设置各类标志
		auto &p = nds[root];
		p.sm = a[i];
	}

	inline void merge_flags(int root) {
		auto &p = nds[root];
		auto &l = nds[lchild(root)];
		auto &r = nds[rchild(root)];
		p.sm = (l.sm + r.sm) % mod * (p.all_mlt) % mod +
						((sum_type)(p.r - p.l) * p.all_add) % mod;
		p.sm %= mod;
	}

	inline void down_mlt_flag(int root) {
		auto &p = nds[root];
		auto &l = nds[lchild(root)];
		auto &r = nds[rchild(root)];
		l.mlt(p.all_mlt, mod);
		r.mlt(p.all_mlt, mod);
		p.all_mlt = 1;
	}

	inline void down_flags(int root) {  // mlt and add
		auto &p = nds[root];
		auto &l = nds[lchild(root)];
		auto &r = nds[rchild(root)];
		l.mlt(p.all_mlt, mod);
		r.mlt(p.all_mlt, mod);
		p.all_mlt = 1;
		l.add(p.all_add, mod);
		r.add(p.all_add, mod);
		p.all_add = 0;
	}

	void build(int l, int r, int root) {
		nds[root].set_basic(l, r);
		if (l + 1 == r) {
			set_flags(root, l);
			return;
		}
		int m = l + (r - l) / 2;
		build(l, m, lchild(root));
		build(m, r, rchild(root));
		merge_flags(root);
	}

	// [l,r)区间的数都加上val
	void add(int l, int r, item_type val, int root = 0) {
		if (l == nds[root].l && r == nds[root].r) {
			nds[root].add(val, mod);
			return;
		}
		down_mlt_flag(root);
		int m = nds[root].mid();
		if (r <= m) {  // only left part
			add(l, r, val, lchild(root));
		} else if (l >= m) {  // only right part
			add(l, r, val, rchild(root));
		} else {
			add(l, m, val, lchild(root));
			add(m, r, val, rchild(root));
		}
		merge_flags(root);
	}

	// [l,r)区间的数都乘以val
	void mlt(int l, int r, item_type val, int root = 0) {
		if (l == nds[root].l && r == nds[root].r) {
			nds[root].mlt(val, mod);
			return;
		}
		down_flags(root);
		int m = nds[root].mid();
		if (r <= m) {  // only left part
			mlt(l, r, val, lchild(root));
		} else if (l >= m) {  // only right part
			mlt(l, r, val, rchild(root));
		} else {
			mlt(l, m, val, lchild(root));
			mlt(m, r, val, rchild(root));
		}
		merge_flags(root);
	}

	sum_type get_sum(int l, int r, int root = 0) {
		if (l == nds[root].l && r == nds[root].r) {
			return nds[root].sm;
		}
		// 勿忘加上all_add lazy标记
		int m = nds[root].mid();
		ll lazy = (nds[root].all_add * (sum_type)(r - l)) % mod;
		ll tmp;
		if (r <= m) {  // only left part
			tmp = get_sum(l, r, lchild(root));
		} else if (l >= m) {  // only right part
			tmp = get_sum(l, r, rchild(root));
		} else {
			tmp = get_sum(l, m, lchild(root))     // left
						+ get_sum(m, r, rchild(root));  // right
		}
		tmp %= mod;
		return (tmp * nds[root].all_mlt % mod + lazy) % mod;
	}

	void out() {
		cout << "n = " << n << "\n";
		for (int i = 0; i < n; ++i) cout << a[i] << " ";
		cout << "\n";
	}
};
};  // namespace lly

lly::segment_tree tr;

int main() {
	// 洛谷 P3373 【模板】线段树 2

	int n, m;
	ll mod;
	// cin>>n>>m;
	scanf("%d%d%lld", &n, &m, &mod);
	tr.n = n;
	tr.mod = mod;
	tr.init();
	tr.build();

	int o, x, y;
	ll k;
	for (int i = 0; i < m; ++i) {
		// cin>>o>>x>>y;
		scanf("%d%d%d", &o, &x, &y);
		--x;
		if (o == 2) {  // [x,y)内都加上k
			scanf("%lld", &k);
			tr.add(x, y, k);
		} else if (o == 3) {  // 询问区间和
			auto sum = tr.get_sum(x, y);
			printf("%lld\n", sum);
		} else {
			scanf("%lld", &k);
			tr.mlt(x, y, k);
		}
	}
	return 0;
}

\end{lstlisting}

\end{document}
